\chapter*{Abstract}

Conflict-free replicated data types (\CRDTs) are a natural structure with which
to communicate information about a shared computation in a distributed setting
where coordination overhead may not be tolerated, and individual participants
are allowed to temporarily diverge from the overall computation.  Within this
setting, there are two classical approaches: state- and operation-based \CRDTs.
The former define a commutative, associative, and idempotent \textit{join}
operation, and their states a \textit{monotone join semi-lattice}. These may be
further distinguished into classical- and $\delta$-state
\CRDTs~\citep{almedia18}. The former communicate their \emph{full} state after
each update, whereas the latter communicate only the \emph{changed} state. On
the other hand, op-based \CRDTs communicate \emph{operations} (not state), thus
making these updates non-idempotent.  Whereas op-based \CRDTs require little
information to be exchanged, they demand relatively strong network guarantees
(exactly-once message delivery), and state-based \CRDTs suffer the opposite
problem. Both satisfy \textit{strong eventual consistency} (\SEC)

We posit that $\delta$-\CRDTs both (1) require minimal communication overhead
from payload size, and (2) tolerate relatively weak network environments, making
them an ideal candidate for real-world use of \CRDTs. Our central intuition is
to note a key similarity between state- and op-based \CRDTs. Namely, that
state-based \CRDTs are equivalent to op-based \CRDTs when the operation is a
join of the state. We extend the previous line of work in \cite{gomes17} by
formalizing this intuition in Isabelle and extending the mechanized proofs
presented there to show that state-based \CRDTs achieve \SEC. We then weaken the
network model in \cite{gomes17} and show that state-based \CRDTs still maintain
\SEC. Finally, we extend our work to gossip members of the \textit{sub-lattice},
thus showing that $\delta$-\CRDTs achieve \SEC also, even under relatively weak
network guarantees.
