\chapter{Conclusion}
\label{chap:conclusion}

In this thesis, we extended the work in~\citet{gomes17} to provide a mechanized
verification that $\delta$-\CRDTs~\citep{almedia18} achieve
\SEC~\citep{shapiro11}.

Our central intuition (c.f., Sections~\ref{sec:state-as-op}
and~\ref{sec:delta-as-op}) was to treat $\delta$-\CRDTs as a refinement of
state-based \CRDTs, which we in turn treat as a refinement of op-based \CRDTs.
This allowed us to successfully verify that two \CRDTs--the G-Counter, and
G-Set--achieve \SEC when specified both in the state- and $\delta$-state based
style.

In addition, we relaxed the network model by removing an assumption that all
messages are unique. While our main result is still predicated on a set of nice
delivery semantics $P$, this allowed us to quantify over an expanded set of all
possible network executions.

Together, this allowed us to restate the main result of~\citet{almedia18} in a
mechanized fashion. We believe that $\delta$-state \CRDTs satisfy an appealing
``best-of-both-worlds'' property. $\delta$-state \CRDTs require relatively
little of the network (like op-based \CRDTs), yet still maintain a relatively
small payload size (like state-based \CRDTs). This places great interest on
formal verification of their convergence properties.

In the future, we hope to see our result extended by specifying $\delta$-state
\CRDTs in terms of their state-based counterparts, as well as mechanizing
well-known anti-entropy algorithms and causality constraints on applying updates
from other replicas~\citep{almedia18}. We believe that this would be sufficient
to remove the precondition on a set of delivery semantics $P$ from our result.
