\documentclass[aspectratio=169,compress]{beamer}

\newcommand{\dcrdt}{\text{$\delta$-CRDT}}

\usepackage{amsmath}
\usepackage{booktabs}
\usepackage{bookmark}
\usepackage{xcolor}

\usepackage{pifont}
\newcommand{\cmark}{\ding{51}}

\usetheme{Copenhagen}
\setbeamercovered{transparent}

\usepackage[backend=biber,style=authoryear,sorting=ynt,natbib=true]{biblatex}
\addbibresource{plse-nov.bib}

\title{Towards verifying $\delta$-state CRDTs}
\author{Taylor Blau}
\institute{PLSE Lunch \\ University of Washington}
\date{November, 2019}

\begin{document}
  \frame{\titlepage}

  \begin{frame}
    \frametitle{Hello}

    \begin{itemize}[<+->]
      \item I'm Taylor; undergraduate working on verifying distributed
        computation.
      \item Used to work on Reincarnate.
      \item I work part-time at GitHub on Git, and will go back full-time in
        2020.
    \end{itemize}
  \end{frame}

  \begin{frame}
    \frametitle{A little overview}

    \begin{itemize}[<+->]
      \item Want to talk about challenges in distributed computation, some
        existing, and new solutions.
      \item Specifically, want to focus on two broad classes of replicated
        datatypes.
      \item Going to talk about the existing work, an important trade-off when
        choosing which kinds of replicated datatype to use, and a promising new
        development.
      \item Will outline research aims, as well as some early results.
      \item Please ask questions, during or after.
    \end{itemize}
  \end{frame}

  \begin{frame}
    \frametitle{Shared/coordinated computation}

    Shared computation requires coordination between otherwise independent units
    of that computation in order to achieve \textit{consensus}. Traditionally,
    this might mean at least one of:

    \begin{itemize}[<+->]
      \item A consensus algorithm, like Paxos or Raft.
      \item Reliable broadcast, where a load-balancer proxies requests to
        multiple nodes in a system.
      \item Leader-follower(s) architecture, with leader election.
      \item Atomic commit.
    \end{itemize}

    \pause

    These are classically difficult to implement correctly. This slide is an
    (out-of-order) list of CSE 452 topics.
  \end{frame}

  \begin{frame}
    \frametitle{Replicated data-types}

    Since~\cite{shapiro11}, Conflict-Free Replicated Datatypes are yet another
    way to solve the coordinated computation problem. They fit when your problem
    is shaped like:
    \begin{enumerate}[<+->]
      \item Cannot tolerate coordination lag (i.e., nodes should be able to
        operate independently of one another, no load balancers required).
      \item Can tolerate temporary divergence.
      \item (Optionally) weak network guarantees; more on this later.
    \end{enumerate}
  \end{frame}

  \begin{frame}
    \frametitle{CRDT overview}

    A \textit{Conflict-free replicated data type} (hereafter, CRDT) is a (1)
    datatype and (2) replication strategy for ensuring that multiple,
    independent, and uncoordinated actors can converge on a shared value.

    \vspace{1em}

    Two broad classes of specification:
    \begin{description}[<+->]
      \item[State-based] Each message is an element in a \textit{monotone join
        semi-lattice} of the possible state-space; replicas define $\sqcup$ to
        take in new states.
        \begin{enumerate}
          \item We will later consider the case where the messages cover a
            subset of the sender's lattice, these are called $\delta$-CRDTs.
        \end{enumerate}
      \item[Op-based] Each message is either an idempotent \textit{query} or a
        mutative operation which is prepared locally, and effected globally.
    \end{description}
  \end{frame}

  \begin{frame}
    \frametitle{State-based CRDT}

    Given a state-space $\mathcal{S}$, a state-based CRDT $\mathcal{P}$ is
    defined~\parencite{shapiro11}:

    \[
      \mathcal{P} = (S, s^0, q, u, m)
    \]
    where:
    \begin{description}
      \item[$S$] the set of all possible states; i.e., that $s^{(t)} \in S$.
      \item[$s^{(0)}$] the initial state.
      \item[$q$] a non-modifying query method.
      \item[$u$] a modifying update method.
      \item[$m$] a \textit{commutative} $\sqcup$ merge operation.
    \end{description}

    \vspace{1em}

    Note that $S$ in fact must form a join semi-lattice; i.e., that if $s, s'
    \in \mathcal{S}$, that:
    \[
      s \sqsubseteq s \sqcup s',\quad s' \sqsubseteq s \sqcup s'
    \]
  \end{frame}

  \begin{frame}
    \frametitle{State-based CRDT}

    Several properties of the lattice $\mathcal{S}$ are exploited to produce an
    appealingly weak set of requirements:
    \begin{enumerate}
      \item A commutative $\sqcup$ allows messages to be delivered out-of-order.
      \item An idempotent $\sqcup$ allows messages to be duplicated.
    \end{enumerate}

    With an appropriate choice of anti-entropy algorithm (e.g., one that gossips
    with other replicas infinitely often), then replicas are guaranteed to
    converge even when messages may be dropped~\citep{shapiro11}.
  \end{frame}

  \begin{frame}
    \frametitle{State-based CRDT example}

    We present the \textit{G-Counter} (read: ``grow-only counter''), which is a
    CRDT implemented as follows~\citep{almeida16}:
    \[
      \mathcal{P} = \left\{\begin{aligned}
        S &= \mathcal{I} \times \mathbb{N} \\
        s^{(0)} &= \left\{ \right\} \\
        q &= \sum_{j \in \mathcal{I}} m(j) \\
        u &= m\left\{ i \mapsto m(i) + 1 \right\} \\
        s \sqcup s' &= \left\{ j \mapsto \max(s(j), s'(j)) : j \in \text{dom}(s)
        \cup \text{dom}(s') \right\} \\
      \end{aligned}\right.
    \]
  \end{frame}

  \begin{frame}
    \frametitle{State-based CRDT example}

    \begin{itemize}
      \item In English, the state-based G-Counter maintains a mapping from host
        identifier to counter value, and answers queries regarding the value of
        the counter as the sum of all of the partial counters.
      \item The state payload can naturally ``drop'' messages, since a sup.
        element in the lattice can easily take its place.
        \begin{itemize}
          \item But, CRDT \textit{composition} can make individual states large
            and complex, leading to as-large payloads, which can be expensive to
            serialize/send/apply/etc.
        \end{itemize}
    \end{itemize}
  \end{frame}

  \begin{frame}
    \frametitle{op-based CRDT}

    Given a state-space $\mathcal{S}$, an op-based CRDT $\mathcal{P}$ is
    defined~\parencite{shapiro11}:

    \[
      \mathcal{P} = (S, s^0, q, t, u)
    \]
    where $S$, $s^0$, and $q$ are the same as in the state-based CRDT, and:
    \begin{description}
      \item[$t$] a non-modifying \textit{prepare-update} method.
      \item[$u$] a modifying \textit{effect-update} method.
    \end{description}

    \[
      \begin{aligned}
        t &: \mathcal{S} \to (\mathcal{S} \to \mathcal{S}) \\
        u &: (\mathcal{S} \to \mathcal{S}) \to \mathcal{S} \to \mathcal{S} \\
      \end{aligned}
    \]

    on some operation $\sigma$, $t$ is run immediately at the replica executing
    $\sigma$, and the resulting effector is applied with $u$ immediately
    thereafter.

    That same effector is then applied at all other replicas by $u$.
  \end{frame}

  \begin{frame}
    \frametitle{op-based CRDT}

    The op-based CRDT has stronger requirements that must be met in order to
    guarantee consistency~\citep{shapiro11} These are a result of a lack of a
    merge method, and thus operations need not (and often don't) form a
    semi-lattice.

    \begin{enumerate}
      \item Operations must be delivered in order.
      \item Operations must not be dropped, lest the entire system be halted.
      \item Operations must be delivered exactly once (or tagged and applied
        exactly once).
    \end{enumerate}

    CRDTs run over a TCP/IP implementation can thusly satisfy the above
    requirements.
  \end{frame}

  \begin{frame}
    \frametitle{op-based CRDT example}

    We now give an example of the op-based G-Counter. The definition is not
    quite as natural as the state-based one above, but is an example of the
    simulate-ability of the two.

    \[
      \mathcal{P} = \left\{\begin{aligned}
        S &= \mathcal{I} \times \mathbb{N} \\
        s^{(0)} &= \left\{ \right\} \\
        q &= \sum_{j \in \mathcal{I}} m(j) \\
        t &= \lambda p.\; p.m\left\{i \mapsto p.m(i) + 1 \right\} \\
        u(f) &= f(m)
      \end{aligned}\right.
    \]

    $t$ is a higher-order function that takes a CRDT instance $p$ and increments
    the partial-counter associated with its own ID. $u$ simply applies that
    function locally.
  \end{frame}

  \begin{frame}
    \frametitle{Where are we?}

    \begin{enumerate}[<+->]
      \item On one end of the spectrum: weak network requirements, payload
        explosion.
      \item On the other end: compact prepare/effect-update messages, strong
        network requirements.
      \item Is there middle ground here?
    \end{enumerate}
  \end{frame}

  \begin{frame}
    \frametitle{$\delta$-CRDT \& semi-lattice interaction}

    In the state-based CRDT example, we note that CRDT programs are sending far
    more data than is involved in their operation:
    \[
      u = m\left\{ i \mapsto m(i) + 1 \right\}
    \]
    Since $\sqcup$ provides that:
    \[
      m' \sqcup u = u,\, \forall m' \sqsubseteq m
    \]

    \pause
    where it would \textit{also} be sufficient to simply send:
    \[
      u = \left\{ i \mapsto m(i) + 1 \right\}
    \]
    where $u$ describes the singleton map mapping $i \in \mathcal{I}$ having
    value one greater than $m(i)$. This idea is due to \cite{almeida16}. Then,
    \[
      \begin{aligned}
        u_i(X) &= X \sqcup u_i^\delta(X) \\
        m\left\{ i \mapsto m(i) + 1 \right\} &= m \sqcup \left\{ i \mapsto m(i) + 1 \right\}
      \end{aligned}
    \]
  \end{frame}

  \begin{frame}
    \frametitle{$\delta$-CRDT \& semi-lattice interaction}

    The idea of applying \textit{$\delta$-mutators} originates
    in~\cite{almeida16}, and is motivated by:
    \begin{enumerate}
      \item $\delta$ states \textit{are} just states; they are merely not quite
        the states you would expect to have.
      \item Several $\delta$ states can be grouped into a
        \textit{$\Delta$-group} and applied efficiently and with fewer overall
        messages.
    \end{enumerate}
  \end{frame}

  \begin{frame}
    \frametitle{$\delta$-CRDT \& semi-lattice interaction}

    Because $\delta$ updates are just states, CRDT programs follow the
    state-based approach and define a commutative and idempotent $\sqcup$ merge
    operation over that lattice and retain \textbf{all of the benefits} of
    the state-based approach \textbf{without the payload size burden}.
  \end{frame}

  \begin{frame}
    \frametitle{Consistency types}

    \begin{description}
      \item[Eventual consistency] An update delivered to one (correct) replica
        is eventually delivered to all replicas~\citep{shapiro11}.
      \item[Strong eventual consistency] (correct) replicas delivered the same
        sequence of updates (eventually) show the same state~\citep{shapiro11}.
      \item[Causal consistency] The causal happens-before relation on operations
        is retained (i.e., maintenance of the visibility
        relation)~\citep{almeida16}.
    \end{description}
  \end{frame}

  \begin{frame}
    \frametitle{Consistency results}

    \footnotetext[1]{Mechanized proof.}
    \footnotetext[2]{Mechanized proof; focused on behavioral aspects.}

    \begin{table}
      \begin{tabular}{rcccccccc}
        & \multicolumn{3}{c}{EC}
        & \multicolumn{3}{c}{SEC}
        & \multicolumn{1}{c}{SC} \\
        \cmidrule(lr){2-4} \cmidrule(lr){5-7} \cmidrule(lr){8-8}
          & state- & op- & $\delta$-
          & state- & op- & $\delta$-
          & $\delta$- \\
        \midrule
          \cite{shapiro11} & - & - & & \cmark & \cmark & & \\
          \cite{almeida16} & & & - & & & \cmark & \cmark   \\
          \cite{gomes17} & & - & & & \cmark\footnotemark[1] & & \\
          \cite{zeller14} & - & & & \cmark\footnotemark[2] & & & \\
          \pause
          Our work? & & & & & & \color{red}{\cmark\footnotemark[1]} & \\
      \end{tabular}
    \end{table}
  \end{frame}

  \begin{frame}
    \frametitle{Other results from~\cite{almeida16}}
    \begin{enumerate}
      \item Original paper to present $\delta$-CRDTs, in which payloads do not
        necessairly belong to the same lattice structure as the interrnal state
        of the CRDT.
      \item $\delta$-CRDTs correspond in execution to state-based CRDTs.
      \item $\Delta$ groups, and algorithms which use $\Delta$ groups in
        \textit{ranges of the event trace} to guarantee causal consistency.
      \item Portfolio of $\delta$-CRDTs, including GSet, 2PSet, PNCounter,
        Lexicographic Counter etc.
      \item Causal $\delta$-CRDTs, dot stores, etc.
    \end{enumerate}
  \end{frame}

  \begin{frame}
    \frametitle{Other results from~\cite{gomes17}}
    \begin{enumerate}
      \item Isabelle/HOL theories about network model, causality, order
        dependence, and so on to build a foundational proof about the
        convergence of op-based CRDTs.
      \item Mechanized proofs for Replicated Growable Array, OR-set, and
        G-counter.
      \item Elegant formalization of Strong Eventual Consistency missing from
        previous literature.
      \item Adequate consideration of \textit{progress} in distributed systems
        in this setting.
    \end{enumerate}
  \end{frame}

  \begin{frame}
    \frametitle{Our research direction}

    \begin{itemize}[<+->]
      \item Show that state-based CRDTs achieve SEC under the network model to
        show that op-based CRDTs achieve SEC.
      \item Fix the above proof, and weaken the network model.
        \begin{itemize}
          \item Want to show a useful result; state-based properties hold
            trivially in an op-based world.
          \item i.e., what is the weakest network model in which a state-based
            system could achieve SEC, and for which no op-based datatype would
            be able to be verified?
        \end{itemize}
      \item Extend our state-based CRDT to a \dcrdt, and show that the results
        still hold.
    \end{itemize}
  \end{frame}

  \begin{frame}
    \frametitle{Early results}

    Key insight: op-based CRDTs are state-based when the operation and state are
    the same.
    \[
      \begin{aligned}
        \textbf{type\_synonym}~('id)~state = {'id} \Rightarrow \text{int option} \\
        \textbf{type\_synonym}~('id)~operation = {'id}~state \\
      \end{aligned}
    \]
    \[
      \begin{aligned}
      \textbf{fun}~\text{gcounter\_op} :: "({'id}~operation) \Rightarrow ({'id}~state) \nrightarrow ({'id}~state)"~\textbf{where} \\
      "\text{gcounter\_op theirs ours} = Some (\lambda x.~\text{option\_max}~(theirs~x)~(ours~x))"
      \end{aligned}
    \]

    Can show that a ``state-based'' CRDT achieves SEC in a network model
    suitable for verification of op-based datatypes.
  \end{frame}

  \begin{frame}
    \frametitle{Open questions}

    \begin{enumerate}
      \item What is the key property of {\dcrdt}s achieving SEC?
      \item What is a suitable network model that shows a useful result for
        \dcrdt-based computation?
      \item Is the path to that model ``through'' the state-based approach?
    \end{enumerate}
  \end{frame}

  \begin{frame}
    \frametitle{Thanks}
    \begin{center}
      \LARGE\it
      Questions?
    \end{center}
  \end{frame}

\end{document}
