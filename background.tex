\chapter{Background}
\label{chap:background}

There are two broad classes of replicated datatypes that achieve strong eventual
consistency through two different mechanisms. The first is called
\textit{state-based} and forms a \textit{lattice} of possible states, where two
states are \textit{joined} together by a join operation defined over that
lattice. The latter is called \textit{operation-} (or \textit{op-})
\textit{based}, and relies on strong message delivery guarantees over a shared
communication channels to communicate updates in such a way that nodes receiving
those updates are guaranteed to execute them identically, so that all
participants in this fashion also reflect the same state. We consider each of
these two kinds in turn:

\section{op-based CRDTs}

\section{State-based CDRTs}
\label{sec:background-state-based-crdts}
A CRDT is \textit{state-based} when the information shared between multiple
participants within the system is \textit{state-centric}; that is, when members
of the system use the announcement of their internal state to drive the system
forward towards the strong form of eventual consistency.

Specifically, a state-based CRDT is a 5-tuple $(S, s^0, q, u, m)$, where $S$ is
the lattice of all possible states, $s^0 \in S$ is the initial state, and $q$,
$u$, $m$ are the query, update, and merge functions~\citep{shapiro11}.
\textit{Querying} a state-based CRDT returns the internal state of that CRDT at
the time it was queried. \textit{Updating} a CRDT updates its internal state to
be some new value, where the kind of update is specific to the particular CRDT.
Finally, \textit{merging} is done when a CRDT wishes to incorporate updates from
some other CRDT into its internal state, so that it might be reflected by a
subsequent query.

We now devote our attention to the \textit{join semi-lattice} $S$ which makes up
all of the possible states reachable in any execution of the CRDT, and which is
central to an understanding of the state-based CRDT's convergence properties.
The semi-lattice $S$ is a partial ordering $\leq$ of all possible states, where
each pair of states has a \textit{least upper-bound} $\sqcup$:
\[
  s = s' \sqcup s'',\quad \mathrm{if}~
    s', s'' \leq s~\mathrm{and}~
    s = \min_{s', s'' \leq s'''} s'''
\]
Recall also that a partial ordering $\leq$ has the following properties, which
are near those of an equivalence class (but do not in practice form an
equivalence class, since partial orderings are antisymmetric):
\begin{description}
  \item[Reflexivity] For all $s \in S$, $s \leq s$.
  \item[Antisymmetry] For all $s, s' \in S$, if $s \leq s'$ and $s' \leq s$,
    then $s = s'$.
  \item[Transitivity] For all $s, s', s'' \in S$, if $s \leq s'$ and $s' \leq
    s''$, then $s \leq s''$.
\end{description}

Each intermediate state in the execution of a state-based CRDT is a member of
the monotone join semi-lattice defined by $S$. States are non-decreasing, that
is, it is always the case that:
\[
  s \sqsubseteq s \sqcup s',\quad \forall s' \in S
\]
This property is used to guarantee that all executions of a state-based CRDT
having received the same set of updates achieve the same final state.

\begin{figure}[H]
  \centering
  \includegraphics[width=.75\textwidth]{figures/1/semi-lattice.pdf}
  \caption{\TODO}
\end{figure}

\TODO This leads to the main result of~\citep{shapiro11}:
\begin{theorem}[Convergence of state-based CRDTs (Shapiro)]
  Assuming that the communication between instances of a state-based CRDT
  satisfying the semi-lattice structure (as above) is eventual and terminating,
  then any state-based object satisfying these conditions is strong eventually
  consistent.
\end{theorem}

\section{Conflict Free Replicated Datatypes}
\subsection{Example: Grow-Only Counter}
\label{sec:example-gcounter}

\subsection{Example: PN-Counter}
\subsection{Example: OR-Set}
\section{Consistency Guarantees}
\subsection{Causal Consistency}
\subsection{Eventual Consistency}
\subsection{Strong Eventual Consistency}
\section{Network Semantics}
