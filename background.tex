\chapter{Background}
\label{chap:background}

This chapter outlines the preliminary information necessary to contextualize the
remainder of this thesis for readers unfamiliar with existing CRDT research.
Here we motivate conflict-free replicated datatypes (CRDTs), formalize state-
and op-based variants of CRDTs, and present examples of common instantiations.
Finally, we conclude with a discussion of the different levels of consistency
guarantees that each CRDT variant offers, and rationalize which levels of
consistency are appealing in certain situations.

\section{Coordinated Replication}
In a distributed system, it is common for more than one participant to need to
have a \textit{view} of the same data. For example, multiple nodes may need to
have access to the same internal data structures necessary to execute some
computation. When a piece of data is shared among many participants in a system,
we say that that data is \textit{replicated}.

However, saying only that some data is ``replicated'' is under-specified. For
example: how often is that data updated among multiple participants? How does
that data behave when multiple participants are modifying it concurrently? Do
all participants always have the same view of the data, or are there temporary
divergences between the participants in the system?

It turns out that the answer of the last question is of paramount importance.
Traditionally speaking, in a distributed system, all participants have an
identical replica of any piece of shared data at all times. That is, at no
moment in time will there be a replica that could atomically compare its
replicated value for some data with any other replica for equality and disagree.
Said otherwise, all replicated values are equal everywhere all at once. This is
an appealing property to say the least, because it allows system designers to
conceptually treat a distributed system as a single unit of computation. That
is, if all replicas maintain the same memory, it is conceptually as if one whole
machine is being replicated many times.

That being said, upholding this requirement is not a straightforward task. An
immediate question arises which is: who coordinates when updates to a piece of
data are replicated to other participants in the system? What happens when the
coordinator becomes unresponsive, or otherwise misbehaves? Who is responsible
for electing a new participant to take over the coordination duties of the
participant which was no longer able to fulfill them?

\section{Distributed Consensus Algorithms}

This gives rise to the area of consensus algorithms. A consensus algorithm,
broadly speaking, is a routine which multiple participants follow in order to
agree on a shared value.

We first state briefly the properties that an algorithm must have to solve
distributed consensus:
\begin{definition}[Distributed Consensus Algorithm, \citep{howard19}]
  \label{def:consensus}
  An algorithm is said to solve distributed consensus if it has the following
  three safety requirements:
  \begin{description}
    \item[Non-triviality] The decided value must have been proposed by a
      participant.
    \item[Safety] Once a value has been decided, no other value will be decided.
    \item[Safe learning] If a participant learns a value, it must learn the
      decided value.
  \end{description}
  In addition, it must satisfy the following two progress requirements:
  \begin{description}
    \item[Progress] Under previously agreed-upon liveness conditions, if a value
      is proposed by a participant, then a value is eventually decided.
    \item[Eventual learning] Under the same conditions as above, if a value is
      decided, then that value must be eventually learned.
  \end{description}
\end{definition}

We summarize the two most popular and often-implemented algorithms for
distributed consensus~\citep{howard20}:

\paragraph{Paxos}
The most popular algorithm in this field is Paxos~\citep{lamport98}. Broadly
speaking, values (corresponding to changing the value of a replicated piece of
data) are \textit{proposed}, \textit{accepted}, and \textit{learned} by
participants in the system. This process is coordinated by an elected
\textit{leader}, which is responsible for communicating with other participants
in order to drive the process forward. When a leader becomes unresponsive, other
participants in the system begin an election process to replace the leader, and
do so when a majority of participants (and the new leader) acknowledge the
  change.

\paragraph{Raft}
Lest we omit another often-implemented consensus algorithm, we briefly discuss
the Raft consensus algorithm as well~\citep{ongaro14}. Raft emphasizes
understandability in its design, and ``separates the key elements of consensus''
by silo-ing replication, leader election, and safety into different sub-problems
at the design level. An execution of Raft consists of several \textit{terms}. To
begin each term, an election is held in order to determine a \textit{leader}.
Once elected, the leader is responsible for disseminating updates to each
replicas copy of the \textit{log}. Conflicting log entries are always resolved
in favor of the leader's copy. Finally, Raft enforces safety by imposing
additional restrictions on the behavior of a term such that the log replication
strategy is proven to be safe. This is argued in~\citep{ongaro14} and
mechanically verified in~\citep{wilcox15}.

\subsection{Safety in Distributed Consensus Algorithms}

Both Raft and Paxos are notoriously difficult to implement correctly in
practice~\citep{howard20}. Distributed consensus algorithms are often the
subject of undergraduate-level courses in networks and distributed systems.
Often, commercially-available implementations of these algorithms are built into
off the shelf solutions, such as Apache Zookeeper and CoreOS's

It is a natural question to ask why these algorithms are so notoriously
difficult to implement in practice. Individually, the properties
in Definition~\ref{def:consensus} seem reasonably in their own right. We propose
that it is the safety property (that once a value has been decided, no other
values will be decided) that makes implementing these algorithms correctly so
difficult in practice. In essence, coordinating the proposals individual
participants submit is the central difficulty of these algorithms.

Conflict-free Replicated Datatypes (CRDTs) are a natural answer to this
question. By allowing participants to temporarily diverge from the state of the
overall computation (c.f., the second property of
Definition~\ref{def:eventual-consistency}), CRDTs allow replicas to violate the
safety property of Definition~\ref{def:consensus}. By giving up the immediacy
and permanence that the safety properties of a traditional distributed consensus
algorithm, CRDTs allow for a dramatically lower implementation burden in
practice, and are substantially easier to reason about.

\section{state-based CRDTs}
\label{sec:state-based-crdts}
We begin with a discussion of state-based CRDTs from their inception
in~\citep{shapario11}. A state-based CRDT is a 5-tuple $(S, s^0, q, u, m)$. An
individual replica of a state-based CRDT is at some state $s^i \in S$ for $i \ge
0$, and is initially $s^0$. The value may be queried by any client or other
replica by invoking $q$. It may be updated with $u$, which has a unique type per
CRDT object. Finally, $m$ merges the state of some other remote replica.

Crucially, the states of a given state-based CRDT form a partially-ordered set
$\langle S, \sqsubseteq \rangle$. This poset is used to form a join
semi-lattice, where any finite subset of elements has a natural least
upper-bound. Consider two elements $s^m, s^n \in S$. The least upper-bound
$s = s^m \sqcup s^n$ is given as:
\[
  \forall s'.\; s' \sqsupseteq s^m, s^n \Rightarrow
    s^m \sqsubseteq s \land
    s^n \sqsubseteq s \land
    s \sqsubseteq s'
\]
In other words, a $s = s^m \sqcup s^n$ is a least upper-bound of $s^m$ and $s^n$
if it is the smallest element that is at least as large as both $s^m$ and $s^n$.

For now, we set aside the query operations $q$, and allow $q$'s implementation
detail that is left to individual CRDTs, and is specific to their use-case. For
e.g., it may be that a vector-counter CRDT's implementation of $q$ is
something like:
\[
  q : S \mapsto \mathbb{N} = \sum_{i \in \mathbb{I}} s(i)
\]
In this example, let $S = \mathbb{N}^{|\mathbb{I}|}_+$, where $\mathbb{I}$ is
the set of replicas, and $s(i)$ returns the $i$th index of the state vector.

We'll focus our attention for now on the implementation of $m$, which is the
function responsible for merging the states of two independent replicas of a
given state-based CRDT. Given a suitable set of states which forms a lattice, we
assume that:
\[
  m(s_1, s_2) = s_1 \sqcup s_2
\]
and that whenever a CRDT replica $r_1$ at state $s_1$ receives an update from
another replica $r_2$ at state $s_2$, that $r_1$ attains a new state $s_1' =
m(s_1, s_2)$. This process, in addition to each replica periodically
broadcasting an update which contains its current state, is carried on
continually, and $m$ is invoked whenever a new state is received. That is, each
replica is evolving over time in response to outside instruction, and in turn
these updates cause internal state transitions, which themselves cause those new
states to be broadcasted and eventually joined at every other replica.

The $\sqcup$ operator has three mathematical properties that make it an
appealing choice for joining states together as in $m$. These are its
\emph{commutativity}, \emph{associativity}, and \emph{idempotency}. That is, for
any states $s_1$, $s_2$, and $s_3$, that:
\begin{itemize}
  \item The operator is \emph{commutative}, i.e., that $s_1 \sqcup s_2 = s_2
    \sqcup s_1$.
  \item The operator is \emph{associative}, i.e., that $s_1 \sqcup (s_2 \sqcup
    s_3) = (s_1 \sqcup s_2) \sqcup s_3$.
  \item Finally, the operator is \emph{idempotent}, i.e., that $s_1 \sqcup s_1 =
    s_1$.
\end{itemize}

These mathematical properties correspond to real-world constraints that often
arise naturally in the area of distributed systems. Take, for example, that
messages may occur out of order. This often happens in, for example, UDP (User
Datagram Protocol) networks, where the received datagrams are not guaranteed to
be in the order that they were sent. Because $\sqcup$ is commutative, replicas
joining the updates of other replicas do not need to receive those updates in
order, because the result of $s_1 \sqcup s_2$ is the same as $s_2 \sqcup s_1$.
That is, it does not matter which of two updates from another replica arrives
first, because the result is the same no matter in which order they are
delivered.

For concreteness, say that we have two replicas, $r_1$ and $r_2$. $r_1$
initially begins at state $s$, and $r_2$ progresses through states $s_1, \ldots,
s_n$ for $n > 0$. We then see that it does not matter the order in which these
updates are delivered to $r_1$. Suppose that we have a mapping $\pi : [n] \to
[n]$ which maps the true order of a state $s_i$ to the order in which it was
delivered. Then, we can see that the choice of $\pi$ is arbitrary, because:
\[
  s \gets s \sqcup (s_{\pi(1)} \sqcup \cdots \sqcup s_{\pi(n)})
\]
for any choice of $\pi$, because
\[
  s_{\pi(1)} \sqcup \cdots \sqcup s_{\pi(n)} = s_1 \sqcup \cdots \cdots s_n
\]
which follows from the fact that $\sqcup$ is commutative, and can be shown
inductively on the number of updates, $n$.

Next, it is often common for packets to be duplicated in transit over a network.
That is, even though a packet may be sent from a source only once, it may be
received by a recipient on the same network multiple times. For this, the
idempotency of $\sqcup$ comes in handy: no matter how many times a state is
broadcast from an evolving replica, any other replica on the network will
tolerate that set of messages, because it only requires the message to be
delivered once. Any additional duplicates are merged in without changing the
state.

Finally, associativity is an appealing property, too, although its applications
are both less immediate and less often-used in this thesis. Suppose that several
replicas of a state-based CRDT reside on a network with, say, high latency, or
it is otherwise undesirable to send more messages on the network than is
necessary. Because associativity implies that the grouping of updates is
arbitrary, a replica can maintain a \textit{set} of pending updates, and
periodically\footnote{``Periodically'' is arbitrary and an implementation
choice, but it would be easy to imagine that this could be interpreted as
whenever the set reaches a certain size, and/or after a certain amount of time
has passed since flushing the set of pending updates.} send that set to other
replicas by first folding $\sqcup$ over it and sending a single update.

Although commutativity and idempotency receive the greatest share of attention
within CRDT research, associativity is most often used in the area of
$\delta$-state CRDTs. We will discuss a couple of applications of the
associative property of $\sqcup$ in detail in Chapter~\ref{chap:future-work},
and overview them briefly now. The associativity of updates means that
$\delta$-state CRDTs means that these objects can form an \emph{delta interval},
say, $\Delta^a_b$, which comprises the updates on the interval of time $[a,b)$.
When certain restrictions on when an interval $\Delta^a_b$ may be flushed based
on the values of $a$, $b$, other properties may be attained such as causal
consistency~\citep{almedia18}. In this thesis, we do not investigate this
particular application further, although it appears to be a promising future
direction.

\section{op-based CRDTs}
\label{sec:op-based-crdts}

Operation-based (op-based) CRDTs, like their state-based counterparts have
internal states that form a semi-lattice. However, their communication styles
differ fundamentally: op-based CRDTs communicate \textit{operations} that
indicate a kind of update to be applied locally, instead of the \textit{result}
of that update (as is the case in state-based CRDTs).

This section discusses the original op-based CRDTs and pays brief attention to
some newer specifications such as \textit{pure op-based CRDTs}~\citep{shapiro11,
baquero17}.

An op-based CRDT is a $6$-tuple $(S, s^0, q, t, u, P)$. As in
Section~\ref{sec:state-based-crdts}, $S$, $s^0$, and $q$, retain their
original meaning (that is, the state set, an initial state, a query function,
and an update function). It is temping to say that the op-based $u$ retains the
same meaning as the state-based $u$, however it does not. In op-based CRDTs,
the pair $(t,u)$ takes the place of the $m$ merging function from state-based
CRDTs. $t$ and $u$ correspond to \textit{prepare-update} and
\textit{effect-update}, respectively. When an update is made by a caller (say,
for example, incrementing the value of an op-based CRDT counter), it is done in
two phases~\citep{shapiro11}:
\begin{enumerate}
  \item First, the \textit{prepare-update} implementation $t$ is applied at the
    replica receiving the update. $t$ is side-effect free, and prepares a
    representation of the operation about to take place.
  \item Next, the \textit{effect-update} implementation $u$ is applied at the
    local replica, causing the desired update to take effect.
  \item Finally, the representation from $t$ is sent to all other replicas using
    the delivery relation $P$, at which point they, too, apply the
    representation of the effect using $u$.
\end{enumerate}

This is the critical distinction between op- and state-based CRDTS:
state-based CRDTs propagate their state by applying a local update and taking
advantage of the lattice structure of their state-space in order to define a
convenient merge function. On the other hand, op-based CRDTs propagate their
state by sending the \textit{representation} of an update to other replicas as
an instruction. This critical juncture translates into a corresponding
relaxation in the operation $(t, u)$, which is that unlike the state-based CRDTs
whose $m$ must be commutative, associative, and idempotent, and op-based CRDT
implementation of $(t, u)$ need only be commutative.

To explain why, we briefly restate the definition of a causal history for
op-based CRDTs:

\begin{definition}[op-based Causal History~\citep{shapiro11}]
An object's casual history $C = \{ c_1, \ldots, c_n \}$ is defined as follows.
Initially, $c_i^0 = \emptyset$ for all $i \in \mathbb{I}$. If the $k$th method
execution is idempotent (that is, it is either $q$ or $t$), then the causal
history remains unchanged in the $k$th step, i.e., that $c_i^k = c_i^{k-1}$. If
the execution at $k$ is non-idempotent (i.e., it is $u$), then $c_i^{k} =
c_i^{k-1} \cup \{ u_i^k(\cdot) \}$.
\end{definition}

Causal history of an op-based CRDT is defined based on the
\textit{happens-before} relation $\to$ as follows. An update $(t,u)$ happens
before $(t',u')$ (i.e., that $(t, u) \to (t', u')$) iff $u \in c_{j}^{K_j(t')}$
if $K_j$ is the injective mapping from operation to execution time. Shapiro and
his co-authors go on to describe a sufficient definition for the commutativity
of $(t,u)$ in op-based CRDTs. In effect, they say that two pairs $(t,u)$ and
$(t',u')$ commute if and only if for any reachable state $s \in S$ the effect of
applying them in either order is the same. That is, $s \circ u \circ u' \equiv s
\circ u' \circ u$.

They claim that having commutativity for concurrent operations as well as an
in-order delivery relation $P$ for comparable updates is sufficient to prove
that op-based CRDTs achieve SEC.

\subsection{Pure op-based CRDTs}

For completeness in this area, we briefly discuss the work of Baquero and his
co-authors in developing \textit{pure} op-based CRDTs~\citep{baquero17}. Pure
op-based CRDTs address a highly-interpretable definition of $t$ and $u$ to
restrict the payload size of the messages that these operations generate and
require, respectively. In particular, they point out that op-based CRDTs may
behave as if they were state-based CRDTs when $t$ encodes the state $s^k$ and
$u$ acts like a full join on the lattice $S$.

A particular advantage of op-based CRDTs over their state-based counter-parts is
that they are generally believed to require less overall bandwidth to
communicate updates. The key trade-off is that op-based CRDTs require a much
stronger delivery semantics $P$ (in particular, $P$ must provide in-order and
at-most-once delivery of messages) in order to attain SEC. Of course, treating
an op-based CRDT as a state-based one eliminates many of these benefits.

To address this issue, Baquero and his co-authors treat the state-space $S$ of
a \emph{pure} op-based CRDT as instead a poset of
operations\footnote{Incomparable items are generally interpreted as concurrent
operations performed on a pure op-based CRDT}, where the state is interpreted as
applying elements in the poset order. They couple this with the additional
restriction that $(t,u)$ can only send operations to other replicas, meaning
that, for instance, they cannot send the entire state of a replica to all other
replicas.

\section{$\delta$-state CRDTs}
In this section, we describe the refinement of CRDTs that is the interest and
focus of the body of this thesis. That is the $\delta$-state CRDT, as described
in~\citep{almedia18}. In their original work, Almeida and his co-authors
describe $\delta$-state CRDTs as:
\begin{quote}
...ship[ping] a \textit{representation of the effect} of recent update operations
on the state, rather than the whole state, while preserving the idempotent
nature of \textit{join}
\end{quote}

We will present an example of the $\delta$-state CRDT in a below section. For
now, we focus on the background material necessary to contextualize
$\delta$-state CRDTs. This refinement can be thought of as taking ideas from
both state- and op-based CRDTs to mediate some of the trade-offs described
above. Like a state-based CRDT, $\delta$-state based CRDTs have both internal
states and message payloads that form a join semi-lattice. This endows the
$\delta$-state CRDT with a commutative, associative, and idempotent \emph{join}
operator, as before. Likewise, this means that the $\delta$-state CRDT supports
weak delivery semantics, such as delayed, dropped\footnote{In this thesis, we
consider dropped messages as having been delayed for an infinite amount of time.
This allows us to reason about a smaller set of delivery semantics.}, reordered,
and duplicated message delivery.

Unlike a state-based CRDT, however, $\delta$-state CRDTs do not send their
internal state $s^k$ after an update at time $k-1$. With sufficient
conditions\footnote{Effectively, sufficient conditions mean an internal state
whose updates can be represented as ``smaller'' items within the set of
reachable states. For example, to communicate the operation ``insert element $x$
into a set'', send the unitary set $\{x\}$ instead of $S \cup \{ x \}$.}, this
means that:
\begin{itemize}
  \item $\delta$-state CRDTs support the same weak requirements from the network
    as ordinary state-based CRDTs. That is, they support dropping, duplicating,
    reordering, and delaying of messages.
  \item $\delta$-state CRDTs have similarly low-overhead of message size as
    op-based CRDTs.
\end{itemize}
On the converse, $\delta$-state CRDTs do not:
\begin{itemize}
  \item ...have potentially large payload size, as state-based CRDTs are prone
    to have.
  \item ...require a strong delivery semantics $P$ that ensures ordered,
    at-most-once delivery as op-based CRDTs do.
\end{itemize}
Said otherwise, $\delta$-state CRDTs have the relative strengths of both state-
and op-based CRDTs without their respective drawbacks. This makes them an area
of interest, and they are the subject to which we dedicate the remainder of this
thesis.

\section{Elementary CRDT instantiations}
In this section, we provide the specification of three common CRDT instantiations
for three of the four aforementioned CRDT refinements. In particular, we show
instantiations of the Grow-Only Counter (G-Counter), Positive-Negative Counter
(PN-Counter), and the Grow-Only Set (G-Set) for state-, op-, and
$\delta$-state-based CRDTs. In each of the below, we assume that $\mathcal{I}$
refers to the set of node identifiers corresponding to the active
replicas\footnote{In this thesis, we consider $\mathbb{I}$ to be fixed during
execution; that is, we do not support addition or deletion of replicas.}. For
now, we assume that this set is fixed.

\subsection{Example: Grow-Only Counter}
\label{sec:example-gcounter}

\begin{figure}[H]
  \centering
  \[
    \textsf{G-Counter}_s = \left\{\begin{aligned}
      S &: \mathbb{N}_+^{|\mathcal{I}|} \\
      s^0 &: \left[ 0, 0, \cdots, 0 \right] \\
      q &: \lambda s. \sum_{i \in \mathcal{I}} s(i) \\
      u &: \lambda s,i. s\left\{ i \mapsto s(i) + 1 \right\} \\
      m &: \lambda s_1, s_2. \left\{ \max\left\{ s_1(i), s_2(i) \right\}: i \in \mathsf{dom}(s_1) \cup
      \mathsf{dom}(s_2) \right\}
    \end{aligned}\right.
  \]
  \caption{state-based \textsf{G-Counter} CRDT}
\end{figure}

\begin{figure}[H]
  \centering
  \[
    \textsf{G-Counter}_o = \left\{\begin{aligned}
      S &: \mathbb{N}_+ \\
      s^0 &: 0 \\
      q &: \lambda s. s \\
      t &: \textsf{inc} \\
      u &: \lambda s,p. s + 1 \\
    \end{aligned}\right.
  \]
  \caption{op-based \textsf{G-Counter} CRDT}
\end{figure}

\begin{figure}[H]
  \centering
  \[
    \textsf{G-Counter}_\delta = \left\{\begin{aligned}
      S &: \mathbb{N}_+^{|\mathcal{I}|} \\
      s^0 &: \left[ 0, 0, \cdots, 0 \right] \\
      q &: \lambda s. \sum_{i \in \mathcal{I}} s(i) \\
      u &: \lambda s,i. \left\{ i \mapsto s(i) + 1 \right\} \\
      m &: \lambda s_1, s_2. \left\{ \max\left\{ s_1(i), s_2(i) \right\}: i \in \mathsf{dom}(s_1) \cup
      \mathsf{dom}(s_2) \right\}
    \end{aligned}\right.
  \]
  \caption{$\delta$-state based \textsf{G-Counter} CRDT}
\end{figure}

\subsection{Example: PN-Counter}
\label{sec:example-pncounter}

\begin{figure}[H]
  \[
    \textsf{PN-Counter}_s = \left\{\begin{aligned}
      S &: \mathbb{N}_+^{|\mathcal{I}|} \times \mathbb{N}_+^{|\mathcal{I}|} \\
      s^0 &: \left[ 0, 0, \cdots, 0 \right] \times \left[ 0, 0, \cdots, 0 \right] \\
      q &: \lambda s. \sum_{i \in \mathcal{I}} (\fst~s)(i) - \sum_{i \in
        \mathcal{I}} (\snd~s)(i) \\
      u &: \lambda s,(i,op). \begin{cases}
             (\fst~s)\left\{ i \mapsto (\fst~s)(i) + 1 \right\} \times (\snd~s) & \text{if $op=+$} \\
             (\fst~s) \times (\snd~s)\left\{ i \mapsto (\snd~s)(i) + 1 \right\} & \text{if $op=-$} \\
           \end{cases} \\
      m &: \lambda s_1, s_2.\, \begin{aligned}
             &\left\{ \max\{ i_1, i_2 \} : i \in \mathsf{dom}((\fst~s_1) \cup (\fst~s_1)) \right\} \\
             \times &\left\{ \max\{ i_1, i_2 \} : i \in \mathsf{dom}((\snd~s_1) \cup (\snd~s_1)) \right\}
           \end{aligned}
    \end{aligned}\right.
  \]
  \caption{state-based \textsf{PN-Counter} CRDT}
\end{figure}

\begin{figure}[H]
  \centering
  \[
    \textsf{PN-Counter}_o = \left\{\begin{aligned}
      S &: \mathbb{N}_+ \times \mathbb{N}_+  \\
      s^0 &: 0 \times 0 \\
      q &: \lambda s. (\fst~s) - (\snd~s) \\
      t &: \lambda i, \mathsf{op}. (i, \mathsf{op}),\quad \mathsf{op} \in \{
        \mathsf{inc}, \mathsf{dec} \} \\
      u &: \lambda s,\mathsf{op}. \begin{cases}
             (\fst~s) + 1 \times (\snd~s) & \text{if $\mathsf{op} = \mathsf{inc}$} \\
             (\fst~s) \times (\snd~s) + 1 & \text{if $\mathsf{op} = \mathsf{dec}$} \\
           \end{cases} \\
    \end{aligned}\right.
  \]
  \caption{op-based \textsf{PN-Counter} CRDT}
\end{figure}

\begin{figure}[H]
  \[
    \textsf{PN-Counter}_\delta = \left\{\begin{aligned}
      S &: \mathbb{N}_+^{|\mathcal{I}|} \times \mathbb{N}_+^{|\mathcal{I}|} \\
      s^0 &: \left[ 0, 0, \cdots, 0 \right] \times \left[ 0, 0, \cdots, 0 \right] \\
      q^\delta &: \lambda s. \sum_{i \in \mathcal{I}} (\fst~s)(i) - \sum_{i \in
        \mathcal{I}} (\snd~s)(i) \\
      u^\delta &: \lambda s,(i,op). \begin{cases}
             \{ i \mapsto 1 \} \times \emptyset & \text{if $op=+$} \\
             \emptyset \times \{ i \mapsto 1 \} & \text{if $op=-$} \\
           \end{cases} \\
      m^\delta &: \lambda s_1, s_2.\, \begin{aligned}
             &\left\{ \max\{ i_1, i_2 \} : i \in \mathsf{dom}((\fst~s_1) \cup (\fst~s_1)) \right\} \\
             \times &\left\{ \max\{ i_1, i_2 \} : i \in \mathsf{dom}((\snd~s_1) \cup (\snd~s_1)) \right\}
           \end{aligned}
    \end{aligned}\right.
  \]
  \caption{$\delta$-state based \textsf{PN-Counter} CRDT}
\end{figure}


\subsection{Example: G-Set}
\label{sec:example-gset}

\TODO[parameterized over a set $\mathcal{X}$]

\begin{figure}[H]
  \centering
  \[
    \textsf{G-Set}_s(\mathcal{X}) = \left\{\begin{aligned}
      S &: \mathcal{P}(\mathcal{X}) \\
      s^0 &: \emptyset \\
      q &: \lambda x. x \in s \\
      u &: \lambda x. s \cup \{ x \} \\
      m &: \lambda s_1, s_2. s_1 \cup s_2 \\
    \end{aligned}\right.
  \]
  \caption{state-based \textsf{G-Set} CRDT}
\end{figure}

\begin{figure}[H]
  \centering
  \[
    \textsf{G-Set}_o(\mathcal{X}) = \left\{\begin{aligned}
      S &: \mathcal{P}(\mathcal{X}) \\
      s^0 &: \emptyset \\
      q &: \lambda x. x \in s \\
      t &: \lambda x. (\textsf{ins}, x) \\
      u &: \lambda p. s \cup (\textsf{snd}~p) \\
      m &: \lambda s_1, s_2. s_1 \cup s_2 \\
    \end{aligned}\right.
  \]
  \caption{op-based \textsf{G-Set} CRDT}
\end{figure}

\begin{figure}[H]
  \centering
  \[
    \textsf{G-Set}_\delta(\mathcal{X}) = \left\{\begin{aligned}
      S &: \mathcal{P}(\mathcal{X}) \\
      s^0 &: \emptyset \\
      q &: \lambda x. x \in s \\
      u &: \lambda x. \{ x \} \\
      m &: \lambda s_1, s_2. s_1 \cup s_2 \\
    \end{aligned}\right.
  \]
  \caption{$\delta$-state based \textsf{G-Set} CRDT}
\end{figure}

\section{Consistency Guarantees}
\subsection{Causal Consistency}
\subsection{Eventual Consistency}
\subsection{Strong Eventual Consistency}
\section{Network Semantics}
