\chapter{Acknowledgements}

This thesis is the product of many ideas grown out of collaboration and
discussion with my advisory committee, as well as other researchers in this
area.

First, Talia Ringer, my senior thesis mentor. Talia's thoughtfulness and
willingness to absorb a new research area was inspiring and encouraged me to
look at this area from a new angle. Her patience in acquainting me with
interactive theorem provers was key in making this thesis possible. Though
always a source of good ideas, this thesis would not exist without Talia's
unwavering support. I would be remiss if I did not mention Talia's encouragement
throughout, even when the process was overwhelming.

Second, Dan Grossman, my faculty advisor. Dan has made my undergraduate
experience meaningful in ways that I am not sure many others are as fortunate as
I to have experienced. Dan took a skeptical pre-freshman, encouraged him to take
CSE 341, and indulged him in many walks back to the Paul G. Allen building after
class. Dan allowed me to T.A. for him, and was unflapped when I informed him
that I had volunteered him to be my faculty advisor. Of course, Dan is also a
fountain of insight, offering new ideas and perspectives when they were needed,
and always giving me something to think about after our meetings.

I would also like to thank Martin Kleppman, as well as his co-authors, for his
constant correspondence throughout this work. His work is foundational to our
approach, and is the basis on which many of our ideas (and proofs) are built.
Martin was always willing to discuss the state of our work, and share new ideas
about alternative network models, communication invariants, and so on.

Finally, I wish to thank my family. My Mom and Dad, for their love, for always
encouraging me, and for giving me the freedom to explore areas that interested
me.  Lastly, I wish to thank Maya Lippard. Maya is my constant source of
inspiration, and without her this thesis would not exist.
