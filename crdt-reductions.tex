\chapter{\CRDT reductions}
\label{chap:crdt-reductions}

This chapter outlines a key component of our proof strategy, which is a pair of
reductions that allow us to view state-based \CRDTs as a refinement of op-based
\CRDTs, and $\delta$-state \CRDTs as a refinement of state-based ones.

Specifically, we will discuss the following:
\begin{itemize}
  \item In Section~\ref{sec:state-as-op}, we will describe a mapping $\phi$ to
    view state-based \CRDTs in terms of their op-based counterparts.
  \item In Section~\ref{sec:delta-as-state}, we will note a key similarity
    between state- and $\delta$-state based \CRDTs and introduce a class
    \emph{generalized state-based \CRDTs}. This intuition will allow us to use
    $\phi$ to reason about $\delta$-state \CRDTs in terms of op-based \CRDTs.
\end{itemize}

In this section, we will identify and discuss the two key intuitions which guide
the remainder of our proof strategy. We'll refer to these intuitions as
``maxims.'' They are stated here in brief, but we will return to them in
Sections~\ref{sec:state-as-op} and~\ref{sec:delta-as-state}.

\begin{maxim}
  \label{maxim:state-as-op}
  A state-based \CRDT is an op-based \CRDT where the \emph{prepare-update} phase
  returns the updated state, and the \emph{effect-update} is a join of two
  states.
\end{maxim}

\begin{maxim}
  \label{maxim:delta-as-state}
  A $\delta$-state based \CRDT is a state-based \CRDT which is allowed to send
  messages which do not reflect the entirety of its current state.
\end{maxim}

\section{state-based \CRDTs as op-based}
\label{sec:state-as-op}
Our discussion begins by addressing the question of how we will use a library
built to verify op-based \CRDTs to verify $\delta$-state \CRDTs. Our approach is
to view the set of $\delta$-state \CRDTs as a refinement of the set of
state-based \CRDTs, which are themselves in turn a refinement of the set of
op-based \CRDTs. In other words, we view op-based \CRDTs as the \emph{most}
expressive of the three categories of \CRDTs that we have discussed so far.
Before discussing the latter correspondence, we'll first begin by relating op-
and state-based \CRDTs to one another.

Consider some state-based \CRDT $C = (S, s^0, q, u, m)$. This object $C$ has a
set of states $S$, an initial state $s^0$, along with functions for querying the
state ($q$), updating its state ($u$), and merging its state with the state of
some other object $(m)$. Our question is to define a mapping $\phi$ as
follows:
\[
  \phi_{\text{state} \to \text{op}} :
    \underbrace{(S, s^0, q, u, m)}_{\text{state-based \CRDTs}} \longrightarrow
    \underbrace{(S, s^0, q, t, u, P)}_{\text{op-based \CRDTs}}
\]
For our purposes, we view $\phi_{\text{state} \to \text{op}}$ as a homomorphism
between state- and op-based \CRDTs. If we had such a structure-preserving
transformation, then we would be able to verify any state-based \CRDT of our
choosing using the framework from~\citet{gomes17} without modification to the
framework. Then, the process for verifying a state-based \CRDT using the
op-based framework looks as follows:
\begin{enumerate}
  \item Take a state-based \CRDT, say $C_s$, of interest for verification.
  \item Convert the state-based \CRDT into an op-based one by computing:
    \[
      C_o = \phi_{\text{state} \to \text{op}}(C_s)
    \]
  \item Use $C_o$ as input to the verification tool.
\end{enumerate}

Note that $P$ (the delivery precondition on the right-hand side) is the only
element which does not have a natural analog on the left-hand side.
Traditionally it is common to have a $P$ which preserves causality, but this is
not necessary for our proofs (since we map states identically as in the
following section). Therefore, we assume that $P$ is always met, in which case
delivery can always occur immediately on the right-hand side.

We'll now turn to describing the details of $\phi_{\text{state} \to \text{op}}$,
which for convenience in this section, we'll abbreviate as simply
$\phi$.\footnote{In the following section, we'll define a new homomorphism
between state- and $\delta$-state based \CRDTs, at which point we will
distinguish between the two mappings when it is unclear which is being referred
to.} To understand $\phi$, we'll consider how it maps the state $S$ (along with
$s^0$ and $q$) separately from how it maps the update procedure $u$.

\subsection{Mapping states under $\phi$}

Let us begin our discussion with a consideration to how $\phi$ maps the state
$S$ from a state-based \CRDT to an op-based one. In practice, it would be
unrealistic to treat the state space of a state-based \CRDT as equal to that of
its op-based counter-part. Doing so would discard one of the key benefits of
op-based \CRDTs over state-based ones, which is that they are often able to
represent the same set of query-able states using simpler structures. For
example, state-based counters (such as the $\textsf{G-Counter}_s$ and
$\textsf{PN-Counter}_s$) often use a vector representation to represent the
number of ``increment'' operations at each node, but op-based counters often
instead use scalars (c.f., the examples in
Section~\label{sec:example-gcounter}).

In order to make $\phi$ a simple reduction, we allow the state spaces of the
\CRDT before and after the reduction to be identical. Though \CRDT designers can
often be more clever than this in practice, this makes reasoning about the
transformation much simpler for the purposes of our proofs.  Likewise, since the
query function $q$ is defined in terms of the state-space, $S$, we let $\phi$
preserve the implementation of $q$ under the mapping, too.

\subsection{Mapping updates under $\phi$}
Now that we have mapped $S$, $s^0$, and $q$ under $\phi$, we still need to
address the implementation of $u$ and $m$ under mapping. Our guiding principle
is the following theorem (which we state and discuss here, but have not
mechanized):
\begin{theorem}
  Let $C_s$ be a state-based \CRDT with $C_s = (S, s^0, q, u, m)$. Define an
  op-based \CRDT $C_o$ as follows:
  \[
    C_0 = \left\{ \begin{aligned}
      S_o &: S \\
      s^0_o &: s^0 \\
      q_o &: q \\
      t_o &: \lambda p.\, u(p...) \\
      u_o &: \lambda s_2.\, m(s^t, s_2) \\
    \end{aligned} \right.
  \]
  then, $C_o$ and $C_s$ reach equivalent states given equivalent updates and
  delivery semantics.
\end{theorem}
\begin{proof}[Proof sketch]
By simulation. Since $s^0_o = s^0$, both objects begin in the same state. Since
$q_o = q$, if the state of $C_o$ and $C_s$ are equal, then $q_o$ will reflect as
much. Finally, an update is prepared locally by computing the updated
state-based representation. That update is applied both locally and at all
replicas by merging the prepared state into $C_o$'s own state, preserving the
equality.
\end{proof}

In other words, we decompose the \textit{update} function of a state-based \CRDT
into the \textit{prepare-update} and \textit{effect-update} functions of an
op-based \CRDT. Let $p$ be the set of parameters used to invoke the update
function $u$ of a state-based \CRDT, i.e., that $u(p...)$ produces the desired
updated state. Then the \textit{prepare-update} returns a serialized
representation of $u(p...)$, which is to say that it returns the updated state.
The \textit{effect-update} implementation then takes that representation and
applies it by invoking the merge function $m$ with the effect representation and
its own state to produce the new state.

This introduces Maxim~\ref{maxim:state-as-op}, which unifies state- and op-based
\CRDTs as behaving identically when the op-based \CRDT performs a join of two
states. We restate this Maxim for clarity:

\setcounter{maxim}{0}
\begin{maxim}
  A state-based \CRDT is an op-based \CRDT where the \emph{prepare-update} phase
  returns the updated state, and the \emph{effect-update} is a join of two
  states.
\end{maxim}

\section{Generalized state-based \CRDTs}
While it would be convenient to be able to map $\delta$-state \CRDTs to
state-based \CRDTs identically, this is not possible. Specifically, it elides
the key benefit of using $\delta$-state \CRDTs over pure state-based \CRDTs,
which is the reduced payload size in $\delta$-state \CRDTs.

To address this, we propose a new \emph{class} of state-based \CRDTs,
encompassing both the state- and $\delta$-state variants, which we refer to as
\emph{generalized state-based \CRDTs}. Like usual, we define a general
state-based \CRDT to be a $5$-tuple $(S, s^0, q, u, m)$, where all elements
except $u$ are defined identically to traditional state-based \CRDTs.

Unlike traditional state-based \CRDTs, however, $u$ does \emph{not} cause the
\CRDT to send its full state after an update. Instead, $u$ returns \emph{any}
state in the lattice $s$ so long as $s \sqsubseteq s^t$, where $s^t$ is the
current state.

Immediately, we can observe that traditional state-based \CRDTs are members of
the class of generalized state-based \CRDTs. This follows from the fact that
after updating to state $s^t$, they send their full state, and $s^t \sqsubseteq
s^t$.

\section{$\delta$-state based \CRDTs as general state-based \CRDTs}
\label{sec:delta-as-state}
Now that we have both a procedure for representing a state-based \CRDT as an
equivalent op-based \CRDT under the mapping $\phi$, and a generalized class of
state-based \CRDTs, we only need to discover a similar such mapping from a
$\delta$-state to generalized state-based \CRDT.  Then, composing the two allows
us to obtain:
\[
  \phi_{\delta \to \text{op}} =
    \phi_{\text{state} \to \text{op}} \circ
    \phi_{\delta \to \text{state}}
\]
Luckily, the mapping $\phi_{\delta \to \text{state}}$ (which we now refer to
as $\phi$ out of convenience, disambiguating where necessary) is identity.
This brings us to Maxim~\ref{maxim:delta-as-state}, which we restate here for
clarity:

\setcounter{maxim}{1}
\begin{maxim}
  All $\delta$-state based \CRDTs are generalized state-based \CRDTs.
\end{maxim}

This is a crucial property of $\delta$-state \CRDTs that we reason about now. Any
$\delta$-state \CRDT has a state-space which describes a lattice, just like
state-based \CRDTs. Likewise, they have query, update, and merge functions. The
only difference between state- and $\delta$-state \CRDTs is that the
\emph{communicated} state is a \emph{fragment} of the overall state. Crucially,
these fragments are themselves part of the lattice $S$, which makes them
suitable for sending around and reasoning about as identically to state-based
\CRDTs, and thus they are generalized state-based \CRDTs.

Therefore, we have a straightforward procedure for reasoning about
$\delta$-state \CRDTs in terms of op-based \CRDTs, which is the following:
\begin{enumerate}
  \item Convert a $\delta$-state \CRDT into a state-based \CRDT by applying
    $\phi_{\delta \to \text{state}}$.
  \item Convert that state-based \CRDT into an op-based \CRDT by applying
    $\phi_{\text{state} \to \text{op}}$.
  \item Reason about the resulting op-based \CRDT using the existing framework
    and implementation.
\end{enumerate}
