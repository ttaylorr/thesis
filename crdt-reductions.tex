\chapter{\CRDT reductions}
\label{chap:crdt-reductions}

This chapter outlines the key component of our proof strategy. We state a
reduction that allows us to covert from $\delta$-state to op-based \CRDTs. This
reduction is used extensively in our mechanized proofs
(Chapter~\ref{chap:example-crdts}). To motivate this reduction's explanation, we
first state an example reduction from pure state-based \CRDTs to op-based
\CRDTs. This reduction is also used in mechanized proofs that examples of
state-based \CRDTs achieve \SEC, but not in the mechanized proofs related to
$\delta$-state based \CRDTs.

Specifically, we will discuss the following:
\begin{itemize}
  \item In Section~\ref{sec:state-as-op}, we will describe a mapping
    $\phi_{\text{state} \to \text{op}}$ to reduce state-based \CRDTs to op-based
    \CRDTs.
  \item In Section~\ref{sec:delta-as-op}, we will describe a mapping
    $\phi_{\delta \to \text{op}}$ to reduce $\delta$-state \CRDTs to op-based
    \CRDTs.
\end{itemize}

We state these reductions as ``maxims''. They are stated here in brief, but we
will return to them in Sections~\ref{sec:state-as-op} and~\ref{sec:delta-as-op}.

\begin{maxim}
  \label{maxim:state-as-op}
  A state-based \CRDT is an op-based \CRDT where the \emph{prepare-update} phase
  returns the updated state, and the \emph{effect-update} is a join of two
  states.
\end{maxim}

\begin{maxim}
  \label{maxim:delta-as-op}
  A $\delta$-state based \CRDT is an op-based \CRDT whose messages are
  $\delta$-fragments, and whose operation is a pseudo-join between the current
  state and the $\delta$ fragment.
\end{maxim}

\section{state-based \CRDTs as op-based}
\label{sec:state-as-op}

This section describes a reduction from state-based \CRDTs to op-based \CRDTs.
We describe this reduction to exemplify how to reduce between \CRDT classes, but
this reduction is ultimately not used in our mechanized proofs, since we do not
reason about state-based \CRDTs explicitly.

Consider some state-based \CRDT $C = (S, s^0, q, u, m)$. This object $C$ has a
set of states $S$, an initial state $s^0$, along with functions for querying the
state ($q$), updating its state ($u$), and merging its state with the state of
some other object $(m)$. Our question is to define a mapping $\phi$ as
follows:
\[
  \phi_{\text{state} \to \text{op}} :
    \underbrace{(S, s^0, q, u, m)}_{\text{state-based \CRDTs}} \longrightarrow
    \underbrace{(S, s^0, q, t, u, P)}_{\text{op-based \CRDTs}}
\]
For our purposes, we view $\phi_{\text{state} \to \text{op}}$ as a homomorphism
between state- and op-based \CRDTs.

Note that $P$ (the delivery precondition on the right-hand side) is the only
element which does not have a natural analog on the left-hand side.
Traditionally it is common to have a $P$ which preserves causality, but this is
not necessary for our proofs (since we map states identically as in the
following section). Therefore, we assume that $P$ is always met, in which case
delivery can always occur immediately on the right-hand side.

We'll now turn to describing the details of $\phi_{\text{state} \to \text{op}}$,
which for convenience in this section, we'll abbreviate as simply
$\phi$.\footnote{In the following section, we'll define a new homomorphism
between op- and $\delta$-state based \CRDTs, at which point we will
distinguish between the two mappings when it is unclear which is being referred
to.} To understand $\phi$, we'll consider how it maps the state $S$ (along with
$s^0$ and $q$) separately from how it maps the update procedure $u$.

\subsection{Mapping states under $\phi$}

Let us begin our discussion with a consideration to how $\phi$ maps the state
$S$ from a state-based \CRDT to an op-based one. In practice, it would be
unrealistic to treat the state space of a state-based \CRDT as equal to that of
its op-based counter-part. Doing so would discard one of the key benefits of
op-based \CRDTs over state-based ones, which is that they are often able to
represent the same set of query-able states using simpler structures. For
example, state-based counters (such as the $\textsf{G-Counter}_s$ and
$\textsf{PN-Counter}_s$) often use a vector representation to represent the
number of ``increment'' operations at each node, but op-based counters often
instead use scalars (cf., the examples in Section~\ref{sec:example-gcounter}).

In order to make $\phi$ a simple reduction, we allow the state spaces of the
\CRDT before and after the reduction to be identical. Though \CRDT designers can
often be more clever than this in practice, this makes reasoning about the
transformation much simpler for the purposes of our proofs. Likewise, since the
query function $q$ is defined in terms of the state-space, $S$, we let $\phi$
preserve the implementation of $q$ under the mapping, too.

\subsection{Mapping updates under $\phi$}
\label{sec:states-under-phi}
Now that we have described the process by which $\phi$ maps $S$, $s^0$, and $q$,
we still need to address the implementation of $u$ and $m$ under mapping. Our
guiding principle is the following theorem (which we state and discuss here, but
have not mechanized):
\begin{theorem}
  Let $C_s$ be a state-based \CRDT with $C_s = (S, s^0, q, u, m)$. Define an
  op-based \CRDT $C_o$ as follows:
  \[
    C_0 = \left\{ \begin{aligned}
      S_o &: S \\
      s^0_o &: s^0 \\
      q_o &: q \\
      t_o &: \lambda p.\, u(p...) \\
      u_o &: \lambda s_2.\, m(s^t, s_2) \\
    \end{aligned} \right.
  \]
  then, $C_o$ and $C_s$ reach equivalent states given equivalent updates and
  delivery semantics.
\end{theorem}
\begin{proof}[Proof sketch]
By simulation. Since $s^0_o = s^0$, both objects begin in the same state. Since
$q_o = q$, if the state of $C_o$ and $C_s$ are equal, then $q_o$ will reflect as
much. Finally, an update is prepared locally by computing the updated
state-based representation. That update is applied both locally and at all
replicas by merging the prepared state into $C_o$'s own state, preserving the
equality.
\end{proof}

In other words, we decompose the \textit{update} function of a state-based \CRDT
into the \textit{prepare-update} and \textit{effect-update} functions of an
op-based \CRDT. Let $p$ be the set of parameters used to invoke the update
function $u$ of a state-based \CRDT, i.e., that $u(p...)$ produces the desired
updated state. Then the \textit{prepare-update} returns a serialized
representation of $u(p...)$, which is to say that it returns the updated state.
The \textit{effect-update} implementation then takes that representation and
applies it by invoking the merge function $m$ with the effect representation and
its own state to produce the new state.

This introduces Maxim~\ref{maxim:state-as-op}, which unifies state- and op-based
\CRDTs as behaving identically when the op-based \CRDT performs a join of two
states. We restate this Maxim for clarity:

\setcounter{maxim}{0}
\begin{maxim}
  A state-based \CRDT is an op-based \CRDT where the \emph{prepare-update} phase
  returns the updated state, and the \emph{effect-update} is a join of two
  states.
\end{maxim}

\section{$\delta$-state based \CRDTs as op-based}
\label{sec:delta-as-op}

In the previous section, we described a general procedure for converting
state-based \CRDTs into op-based \CRDTs. In this section, we treat the insight
from the previous section as guidance for how to design a similar reduction to
convert $\delta$-state \CRDTs into op-based \CRDTs. We will use this reduction
to encode $\delta$-state \CRDTs into the library presented in~\citet{gomes17} in
order to verify that $\delta$-state \CRDTs are \SEC.

Similarly as in the previous section, we describe a (new) mapping $\phi_{\delta
\to \text{op}}$ of type:
\[
  \phi_{\delta \to \text{op}} :
    \underbrace{(S, s^0, q, u^\delta, m^\delta)}_{\text{$\delta$-based \CRDTs}}
    \longrightarrow
    \underbrace{(S, s^0, q, t, u, P)}_{\text{op-based \CRDTs}}
\]

For the same reasons as in Section~\ref{sec:states-under-phi}, we let $\phi$
preserve the state space, initial state, and query function. Again, we let $P$
be the delivery precondition which is always met (since messages exchanged are
idempotent, and so there is no need to preserve either causality or at-most-once
delivery as is traditional).

In Section~\ref{sec:state-as-op}, we treated a state-based \CRDT's state as
the representation of the effect for an operation-based \CRDT. In this section,
we do the same for the \emph{$\delta$-state fragment}, which we naturally think
as a difference of two states.

Concretely, let $t : S \to S \to T$ for some type $T$ not necessarily equal to
$S$ which represents the type of all $\delta$-fragments. We define two examples
as follows:
\begin{itemize}
  \item For the G-Set \CRDT, the $\delta$ mutator, $m^\delta$ produces the
    singleton set containing the element added in the last operation. Since only
    one item can be added at a time, computing the following with the before-
    and after-states is sufficient to generate the representation:
    \[
      t = \lambda s_1, s_2.\, s_2 \setminus s_1
    \]
    where $S = T = \mathcal{P}(\mathcal{X})$. This is an example where the \CRDT
    has a type where both the state- and $\delta$-state fragments are members of
    $S$.
  \item For the G-Counter \CRDT, the $\delta$ mutator produces a pair
    type containing the identifier of a node with a changed value, and the new
    value which is assigned to that identifier. $t$ is defined as:
    \[
      t = \lambda s_1, s_2.\, \min_{\substack{i \in \mathcal{I} \\ s_1[i] \ne
        s_2[i]}} (i, s_2[i])
    \]
    (Observe that this function is not defined for two states $s_1 = s_2$, nor
    does it need to be, since the before- and after states are guaranteed to be
    different after invoking $u^\delta$).

    Here we have an example of $T \ne S$, where $T$ instead equals
    $\isacharprime\isa{id} \times \mathbb{N}$.
\end{itemize}

$t$ is now capable of generating the $\delta$-state fragment corresponding to
any pair of states from before and after and invocation of $u^\delta$. Now we
need to define the op-based \CRDT's implementation of $u$ to recover a new state
given a value of type $T$. Here, let $u : S \to T \to S$, which takes in a
current state as well as a $\delta$-fragment and produces a new state.

Intuitively, $u$ is a sort of inverse over the last argument and return value of
$t$. That is, where $t$ was taking the difference of two states, $u$ recovers
that difference into a new state. We define two example implementations of $u$
as follows:
\begin{itemize}
  \item For the G-Set \CRDT, the new state is recovered by taking the union of
    the current state, along with the state carrying the new item. That is:
    \[
      u = \lambda s, t.\, s \cup t
    \]
  \item For the G-Counter \CRDT, the new state is recovered by taking the old
    state, and replacing the entry whose index is equal to the first part of an
    update with the value described by the second part of that update.
\end{itemize}

Importantly, $u$ and $t$ needs to satisfy three important properties:
\begin{enumerate}
  \item $u$ and $t$ can never work together to produce a state which is not
    by either the current state, or the $\delta$-fragment. That is, for any
    state $s'$, we must have that:
    \[
      \forall s \sqsubseteq s'.\, u(s, t(s, s')) \sqsubseteq s'
    \]
    Or in other words, if our starting state is lower in the lattice than $s'$,
    taking the $\delta$-fragment between $s$ and $s'$ and then re-applying that
    to $s$ cannot produce a new state which is $\sqsupseteq s'$.
  \item At all times, all replicas must reflect all updates performed at that
    replica.
  \item All replicas which have received the same set of messages have the same
    state.
\end{enumerate}

Together, these properties are sufficient to re-introduce
Maxim~\ref{maxim:delta-as-op}, which we restate here for clarity:
\setcounter{maxim}{1}
\begin{maxim}
  A $\delta$-state based \CRDT is an op-based \CRDT whose messages are
  $\delta$-fragments, and whose operation is a pseudo-join between the current
  state, and the $\delta$ fragment.
\end{maxim}

Therefore, we have a straightforward procedure for reasoning about
$\delta$-state \CRDTs in terms of op-based \CRDTs, which is to convert any
$\delta$-state \CRDT into an op-based \CRDT, and then use the existing framework
of~\citet{gomes17} to mechanize that that \CRDT achieves \SEC.
