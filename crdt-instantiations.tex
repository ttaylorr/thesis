\chapter{Elementary \CRDT instantiations}
\label{chap:crdt-instantiations}

In this chapter, we provide the specification of two common \CRDT instantiations
in an op-, state-, and $\delta$-state based style. We discuss the Grow-Only
Counter (G-Counter) and Grow-Only Set (G-Set). Both of these will be the subject
of our verification efforts in Chapter~\ref{chap:example-crdts}.

In each of the below, we assume that $\mathcal{I}$ refers to the set of node
identifiers corresponding to the active replicas. In this thesis, we
consider $\mathcal{I}$ to be fixed during execution; that is, we do not support
addition or deletion of replicas. In practice, \CRDTs do support a dynamic set
of replicas, but we make this assumption for the simplicity of our formalism.

\section{Example: Grow-Only Counter}
\label{sec:example-gcounter}

\subsection{State-based G-Counter}
The G-Counter supports two very simple operations: \textsf{inc} (increment), and
query. When \textsf{inc} is invoked, the counter updates its internal state to
increment the queried value by one. When query is invoked, the counter returns a
number which represents the number of increment operations that have occurred
globally in the system, for which the replica processing the query knows about.
Note that this number is always at least as large as the number of times that
\textsf{inc} has been invoked \textit{at that replica}, and never larger than
the true value of times \textsf{inc} has been invoked globally.

This is our first example of \SEC, where replicas that are ``behind,'' i.e.,
that have not received all updates from all other replicas, are not guaranteed
to reflect the same value upon being queried.\footnote{Perhaps these messages
were delayed or dropped in transit, or otherwise the other replicas have not
broadcast their updates yet. The latter is uncommon in traditional state-based
\CRDTs, but is an often-used operation in variants of state-based \CRDTs
(including $\delta$-state \CRDTs) where updates are bundled into
\emph{intervals} which are sent in a way that preserves causality of updates.}
Concretely, suppose that an \textsf{inc} has occurred at at least one other
replica which has not yet broadcast its updated state. The replica being queried
will have therefore not yet merged the updated state from the replica(s)
receiving \textsf{inc},\footnote{Because we cannot merge updates we do not know
about.} and so those update(s) will not be reflected in the value returned by
querying.

We present a state-based G-Counter \CRDT for concreteness, and then discuss its
definition:

\begin{figure}[H]
  \centering
  \[
    \textsf{G-Counter}_s = \left\{\begin{aligned}
      S &: \mathbb{N}_0^{|\mathcal{I}|} \\
      s^0 &: \left[ 0, 0, \cdots, 0 \right] \\
      q &: \lambda s.\, \sum_{i \in \mathcal{I}} s(i) \\
      u &: \lambda s,i.\, s\left\{ i \mapsto s(i) + 1 \right\} \\
      m &: \lambda s_1, s_2.\, \left[ \max\left\{ s_1(i), s_2(i) \right\}: i \in \mathsf{dom}(s_1) \cup
      \mathsf{dom}(s_2) \right]
    \end{aligned}\right.
  \]
  \caption{Specification of a state-based \textsf{G-Counter} \CRDT.}
  \label{fig:state-gcounter}
\end{figure}

Notice that the state space $\mathbb{N}^{|\mathcal{I}|}_0$ does not match the
return type of the query function, $q$, which is simply $\mathbb{N}_0$. In
Figure~\ref{fig:state-gcounter}, we utilize a \emph{vector} counter, which
should be familiar to readers acquainted with \emph{vector
clocks}~\citep{lamport78}.\footnote{Unlike traditional vector clocks, the vector
\emph{counter} only stores in each replica's slot the number of \textsf{inc}
operations performed \emph{at that replica}.}

\begin{figure}[H]
  \centering
  \includegraphics[width=.6\textwidth]{figures/vector-state.pdf}
    \caption{A correct execution of vector-based state G-Counters exchanging
      updates.}
\end{figure}

When an \textsf{inc} is invoked at the $i$th replica, it updates its own state
to increment by one the vector element associated with the $i$th replica, here
denoted $s\{i \mapsto s(i) + 1\}$. Finally, upon receiving an update from
another replica, the pair-wise maximum is taken on each of the vector elements.
Note that this is a commutative, associative, and idempotent operation, and so
it forms the least upper-bound of a lattice of vectors of natural numbers.

\subsection{op-based G-Counter}

In the op-based variant of the G-Counter, we can
rely on a delivery semantics $P$ which guarantees at-most-once message
delivery.\footnote{That is, the network is allowed to drop, reorder, and delay
messages, but a single message will never be delivered more than once.} From
this, we say that replicas which are ``behind'' have not yet received the set of
all \textsf{inc} operations performed at other replicas. Replicas which are
``behind'' may ``catch up'' when they receive the set of undelivered messages.
However, these replicas never are ``ahead'' of any other replica, i.e., they
never receive a message which doesn't correspond to a single \textsf{inc}
operation at some other replica, thus they need not be idempotent.

We present now the full definition of the op-based G-Counter:

\begin{figure}[H]
  \centering
  \[
    \textsf{G-Counter}_o = \left\{\begin{aligned}
      S &: \mathbb{N}_0 \\
      s^0 &: 0 \\
      q &: \lambda s.\, s \\
      t &: \textsf{inc} \\
      u &: \lambda s,p.\, s + 1 \\
    \end{aligned}\right.
  \]
  \caption{Specification of an op-based \textsf{G-Counter} \CRDT.}
\end{figure}

Because replicas are sometimes behind but never ahead, we know that the number
of messages received at any given replica is no greater than the sum of the
number of \textsf{inc} operations performed at other replicas, and the number of
\textsf{inc} operations performed locally. So, the op-based G-Counter needs only
to keep track of the number of \textsf{inc} operations it knows about globally,
and this can be done using a single natural number. Hence, $S = \mathbb{N}_0$,
and the bottom state is $0$.

The query operation $q$ is as straightforward as returning the current state.
The \emph{prepare-update} function $t$ always produces the sentinel
\textsf{inc}, indicating that an increment operation should be performed at the
receiving replica. Finally, $u$ takes a state and an arbitrary
payload\footnote{Unused in the implementation here, since the only operation is
\textsf{inc}.} and returns the successor.

Another approach to specifying the op-based G-Counter \CRDT would be to more
closely mirror the state-space of its state-based counterpart, as follows:

\begin{figure}[H]
  \centering
  \[
    \textsf{G-Counter}_o' = \left\{\begin{aligned}
      S &: \mathbb{N}_0^{|\mathcal{I}|} \\
      s^0 &: [ 0, 0, \cdots, 0 ] \\
      q &: \lambda s.\, \sum_{i \in \mathcal{I}} s(i) \\
      t &: (\textsf{inc}, i)) \\
      u &: \lambda s,p.\, s\{ i \mapsto s(i) + 1 \} \\
    \end{aligned}\right.
  \]
  \caption{Alternative specification of an op-based \textsf{G-Counter} \CRDT.}
\end{figure}

where $i$ represents the local node's identifier. Note that, while correct,
restrictive delivery semantics $P$ do not require such a verbose specification,
since the at-most-once delivery guarantees allow us to simply increment our
local count each time we receive an update, since no updates are duplicated over
the network.

\subsection{$\delta$-state based G-Counter}
We conclude this subsection by turning our attention to the $\delta$-state based
G-Counter. We begin first by presenting its full definition:

\begin{figure}[H]
  \centering
  \[
    \textsf{G-Counter}_\delta = \left\{\begin{aligned}
      S &: \mathbb{N}_0^{|\mathcal{I}|} \\
      s^0 &: \left[ 0, 0, \cdots, 0 \right] \\
      q^\delta &: \lambda s.\, \sum_{i \in \mathcal{I}} s(i) \\
      u^\delta &: \lambda s,i.\, \left\{ i \mapsto s(i) + 1 \right\} \\
      m^\delta &: \lambda s_1, s_2.\, \left\{ \max\left\{ s_1(i), s_2(i) \right\}: i \in \mathsf{dom}(s_1) \cup
      \mathsf{dom}(s_2) \right\}
    \end{aligned}\right.
  \]
  \caption{Specification of a $\delta$-state based \textsf{G-Counter} \CRDT.}
\end{figure}

It is worth mentioning the extreme levels of similarity it shares with its
state-based counterpart. Like the state-based G-Counter, the $\delta$-state
based G-Counter uses the state-space $\mathbb{N}^{|\mathcal{I}|}_0$, and has
$s^0 = [0, 0, \cdots, 0]$. Its query operation and merge are defined
identically.

However, unlike the state-based G-Counter, the $\delta$-state based G-Counter
implements the update function as $\lambda s,i.\, \{ i \mapsto s(i) + 1\}$. That
is, instead of returning the amended map (recall: $s\{ \cdots \}$), the
$\delta$-state based G-Counter returns the \emph{singleton map} containing
\emph{only} the updated index. Because of the definition of $m$ (namely, that it
does a pairwise maximum over the \emph{union} of the domains of the two states),
sending the singleton map is equivalent to sending the full map with all other
entries being equal.

\begin{figure}[H]
  \centering
  \includegraphics[width=.7\textwidth]{figures/vector-delta.pdf}
  \caption{A pair of vector-based $\delta$-state G-Counters replicas exchanging
    updates with each other.}
\end{figure}

This follows from the facts that: (1) the entry being updated has the same
pairwise maximum independent of all other entries in the map, and (2) the
pairwise maximum of all \emph{other} entries does not depend on the updated
entry. So, taking the pairwise maximum of any state with the singleton map
containing one updated value is equivalent to taking the pairwise maximum with
our own state modulo one updated value. $m$ is therefore referred to as a
\emph{$\delta$-mutator}, and the value it returns is an \emph{$\delta$
mutation}~\citep{almedia18}.

This principle of sending \emph{smaller} states (the $\delta$ mutations) which
communicate only the \emph{changed} information is a general principle which
we will return to in the remaining example.

\section{Example: G-Set}
\label{sec:example-gset}

The G-Set is the other primitive \CRDT that we study in this thesis. In essence,
the G-Set is a \emph{monotonic set}. In other words, the G-Set supports the
insertion and query operations, but does not support item removal. This
is a natural consequence of the state needing to form a monotone semi-lattice,
where set deletion would destroy the lattice structure.\footnote{To support
removal from a \CRDT-backed set, the 2P-Set is often used. Verifying this object
is left to future work, which we discuss in
Section~\ref{sec:future-pair-locale}.}

\subsection{State-based G-Set}

We begin our discussion with the state-based G-Set \CRDT, the definition of
which we present below. This is our first example of a \emph{parametric} \CRDT
instance, where the type of the \CRDT is defined in terms of the underlying set
of items that it supports.

\begin{figure}[H]
  \centering
  \[
    \textsf{G-Set}_s(\mathcal{X}) = \left\{\begin{aligned}
      S &: \mathcal{P}(\mathcal{X}) \\
      s^0 &: \{ \} \\
      q &: \lambda x.\, x \in s \\
      u &: \lambda x.\, s \cup \{ x \} \\
      m &: \lambda s_1, s_2.\, s_1 \cup s_2 \\
    \end{aligned}\right.
  \]
  \caption{state-based \textsf{G-Set} \CRDT}
  \label{fig:gset-state}
\end{figure}

For some set $\mathcal{X}$, we can consider the state-based G-Set \CRDT
instantiated over it, $\textsf{G-Set}_s(\mathcal{X})$. The state-space of this
\CRDT is the power set of $\mathcal{X}$, which we denote
$\mathcal{P}(\mathcal{X})$. Initially, the G-Set begins as the empty set, here
denoted $\{ \}$. The three operations are defined as follows:
\begin{itemize}
  \item The query function $q$ is an unary relation, i.e., it determines which
    elements are contained in the G-Set.
  \item The update function $u$ produces the updated set formed by taking the
    union of the existing set, and the singleton set containing the item
    to-be-added.
  \item Finally, the merge function $m$ takes the union of two sets.
\end{itemize}
Note crucially that the merge function $\cup$ defines the least upper-bound of
two sets, and thus endows our \CRDT with a lattice structure. In this lattice of
sets, we say that for some set $\mathcal{X}$, the lattice formed is $\langle
\mathcal{P}(\mathcal{X}), \subseteq \rangle$.

\subsection{op-based G-Set}
In the op-based variant of the G-Set \CRDT, we replace the state-based \CRDT's
update function $u$ with the op-based pair $(t,u)$. The state space, initial
state, as well as the query and merge functions ($q$ and $m$, respectively) are
defined identically. We present the full definition as follows:

\begin{figure}[H]
  \centering
  \[
    \textsf{G-Set}_o(\mathcal{X}) = \left\{\begin{aligned}
      S &: \mathcal{P}(\mathcal{X}) \\
      s^0 &: \{ \} \\
      q &: \lambda x.\, x \in s \\
      t &: \lambda x.\, (\textsf{ins}, x) \\
      u &: \lambda p.\, s \cup \{(\textsf{snd}~p)\} \\
      m &: \lambda s_1, s_2.\, s_1 \cup s_2 \\
    \end{aligned}\right.
  \]
  \caption{op-based \textsf{G-Set} \CRDT}
\end{figure}

The only difference between this \CRDT instantiation and the state-based one is
in the definition of $(t,u)$.\footnote{This is a pattern that will become
familiar during Chapter~\ref{chap:example-crdts}.} In the state-based \CRDT, we
sent the updated state, i.e., $s \cup \{ x \}$. In the op-based variant, we send
a \emph{representation} of the effect, which we take to be the pair
$(\textsf{ins}, x)$, where $\textsf{ins}$ is a sentinel marker indicating that
the second element in the pair should be inserted.

Upon receipt of the message $(\textsf{ins}, x)$, our op-based G-Set \CRDT
computes the new state $s \cup \{ (\snd~p) \}$, where $p$ is the message
payload.

\subsection{$\delta$-state based G-Set}
Finally, we turn our attention to the $\delta$-state based G-Set \CRDT. As was
the case with the $\delta$-state based G-Counter \CRDT, this object is defined
identically as to the state-based counter, with the notable exception of its
update function, $u$.\footnote{This again will be another familiar pattern in
Chapter~\ref{chap:example-crdts}.}

For full formality, we present its definition below:

\begin{figure}[H]
  \centering
  \[
    \textsf{G-Set}_\delta(\mathcal{X}) = \left\{\begin{aligned}
      S &: \mathcal{P}(\mathcal{X}) \\
      s^0 &: \{ \} \\
      q^\delta &: \lambda x.\, x \in s \\
      u^\delta &: \lambda x.\, \{ x \} \\
      m^\delta &: \lambda s_1, s_2.\, s_1 \cup s_2 \\
    \end{aligned}\right.
  \]
  \caption{$\delta$-state based \textsf{G-Set} \CRDT}
\end{figure}

Here, the only difference is between the state- and $\delta$-state based \CRDT's
definition of the update method, $u$. In the state-based G-Set, update was
defined as $u : \lambda x.\, s \cup \{ x \}$. But in the $\delta$-state based
G-Set, the update is defined as $u : \lambda x.\, \{ x \}$. Note crucially that
these two kinds of updates are equal when applied to the same local state.
Consider a state- and $\delta$-state based G-Set, both starting at the same
state $s^t$. For the state-based G-Set, we have:
\[
  \begin{aligned}
    m(s^t, u(x))
      &= s^t \cup (s^t \cup \{ x \}) \\
      &= (s^t \cup s^t) \cup \{ x \} \\
      &= s^t \cup \{ x \} \\
  \end{aligned}
\]
whereas for the $\delta$-based G-Set, we have directly:
\[
  m^\delta(s^t, u^\delta(x)) = s^t \cup \{ x \}
\]
