\chapter{Future Work}
\label{chap:future-work}

This chapter outlines potential future research directions based on interesting
and under-explored areas in this work. Here, we will outline six directions in
the area of verifying $\delta$-state \CRDTs, as well as some insight that might
be gained by exploring each of these directions. It is our hope that future
researchers in this area may choose to conduct further investigation into these
areas.

\section{Verifying additional $\delta$-state \CRDTs}
In our work, we presented examples of two $\delta$-state \CRDTs: the $\delta$
G-Counter, and the $\delta$ G-Set. An immediate future direction is to
investigate and verify more instances of $\delta$-state CRDTs.

One area of particular interest is in the \emph{composition} of multiple
$\delta$-state CRDTs. We have begun investigating the instantiation of a $pair$
locale, which takes as arguments two independent $\delta$-state \CRDTs, known as
``left'' and ``right''. Our hope is that provided existing instantiations of
both of the sub-\CRDTs, that a $pair$ locale given two already-verified \CRDTs
could be used without additional proof burden to create another instance of the
$network{\isacharunderscore}with{\isacharunderscore}ops$ locale. That is: can
two already-verified $\delta$-state \CRDTs be used to compose a new
$\delta$-state CRDT which is their product without additional proof burden?

If this were possible, two new \CRDTs would be verified without effort: the
PN-Counter and the 2P-Set.

\begin{enumerate}
  \item More \CRDTs, including comparable pair types.
  \item Delta intervals, either by simulation (e.g., setting the operation to be
    a list of operations), or more directly.
  \item Anti-entropy algorithms.
  \item Other proof techniques, i.e., by simulation to state-based \CRDTs in a
    $\delta$-state-first system.
  \item End-to-end verification.
  \item Pure $\delta$-state \CRDTs.
\end{enumerate}
