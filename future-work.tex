\chapter{Future Work}
\label{chap:future-work}

This chapter outlines potential future research directions based on interesting
and under-explored areas in this work. Here, we will outline six directions in
the area of verifying $\delta$-state \CRDTs, as well as some insight that might
be gained by exploring each of these directions. It is our hope that future
researchers in this area may choose to conduct further investigation into these
areas.

\section{Verifying additional $\delta$-state \CRDTs}
\label{sec:future-pair-locale}
In our work, we presented examples of two $\delta$-state \CRDTs: the $\delta$
G-Counter, and the $\delta$ G-Set. An immediate future direction is to
investigate and verify more instances of $\delta$-state CRDTs.

One area of particular interest is in the \emph{composition} of multiple
$\delta$-state CRDTs. We have begun investigating the instantiation of a $pair$
locale, which takes as arguments two independent $\delta$-state \CRDTs, known as
``left'' and ``right''. Our hope is that provided existing instantiations of
both of the sub-\CRDTs, that a $pair$ locale given two already-verified \CRDTs
could be used without additional proof burden to create another instance of the
$network{\isacharunderscore}with{\isacharunderscore}ops$ locale. That is: can
two already-verified $\delta$-state \CRDTs be used to compose a new
$\delta$-state CRDT which is their product without additional proof burden?

If this were possible, two new \CRDTs would be verified without effort: the
PN-Counter and the 2P-Set. These two \CRDTs are the most straightforward
composition of other known \CRDTs. Namely, the PN-Counter supports both an
\textsf{inc} and \textsf{dec} operation by maintaining \emph{two} counters (each
of which is treated as a single G-Counter, so the overall state is still
monotone and thus forms a join semi-lattice).

The PN-Counter has two $\delta$-state based G-Counter, which we refer to as
$(\fst s)$ and $(\snd s)$, where $s \in S$ refers to the state of the
PN-Counter. One possible specification for a $\delta$-based PN-Counter follows:

\begin{figure}[H]
  \[
    \textsf{PN-Counter}_\delta = \left\{\begin{aligned}
      S &: \mathbb{N}_+^{|\mathcal{I}|} \times \mathbb{N}_+^{|\mathcal{I}|} \\
      s^0 &: \left[ 0, 0, \cdots, 0 \right] \times \left[ 0, 0, \cdots, 0 \right] \\
      q^\delta &: \lambda s. \sum_{i \in \mathcal{I}} (\fst~s)(i) - \sum_{i \in
        \mathcal{I}} (\snd~s)(i) \\
      u^\delta &: \lambda s,(i,op). \begin{cases}
             \{ i \mapsto 1 \} \times \emptyset & \text{if $op=+$} \\
             \emptyset \times \{ i \mapsto 1 \} & \text{if $op=-$} \\
           \end{cases} \\
      m^\delta &: \lambda s_1, s_2.\, \begin{aligned}
             &\left\{ \max\{ i_1, i_2 \} : i \in \mathsf{dom}((\fst~s_1) \cup (\fst~s_1)) \right\} \\
             \times &\left\{ \max\{ i_1, i_2 \} : i \in \mathsf{dom}((\snd~s_1) \cup (\snd~s_1)) \right\}
           \end{aligned}
    \end{aligned}\right.
  \]
  \caption{$\delta$-state based \textsf{PN-Counter} \CRDT}
\end{figure}

Minor additional consideration is given to the updating function,
$u^\delta$, which returns an empty-set on the counter \emph{not} being updated.
Finally, the merging function $m^\delta$ merges the left- and right-hand sides
of the counter separately, and returns a pair. The 2P-Set is similar in function
to the above, substituting a $\delta$-state based G-Set in place of the
G-Counter.

If such a $\isa{pair}$ locale exists, we believe it would be as straightforward
as instantiating this locale over two copies of the G-Counter and G-Set to
obtain the PN-Counter and 2P-Set immediately.

\section{Direct $\delta$-state \CRDT proofs}

To explore this idea, we drew significant inspiration from the work of Almedia
and his co-authors in~\citep{almedia18} to restate $\delta$-state \CRDTs in
terms of op-based \CRDTs in an effort to reuse as much of their library as
possible.

A significant drawback of this approach is that we are bound to the same
restrictions as op-based \CRDTs, which are inherently more restricted than
state-based \CRDTs. Much of this restriction comes from the \emph{eventual
delivery} property of \EC, which states that~\citep{shapiro11}:
\[
  \forall i, j.\, f \in c_i \Rightarrow \lozenge f \in c_j
\]
or that for any pair of correct replicas $i$, $j$ with histories $c_i$ and
$c_j$, respectively, an update received at one of those replicas is eventually
received at all other replicas.

Of course, under weak delivery semantics (i.e., in the case that the network may
delay messages for an infinite amount of time), op-based \CRDTs do not achieve
this property~\citep{shapiro11}. Namely, if an operation is performed at some
replica, and that message is dropped while in transit to another replica, that
replica will never receive the message.

State-based \CRDTs do not suffer from this problem, since \emph{every} update
they send encapsulates the history of all previous updates, since each update is
either reflected in the state, or subsumed by some later update which is itself
reflected in the state~\citep{shapiro11}. Since the entirety of the state is
shared with each replica during an update, state-based \CRDTs do not need to
impose an additional delivery relation in order to prove that they achieve \SEC.

op-based \CRDTs, on the other hand, \emph{do} need to specify an additional
delivery relation on top of their definition. That is, the delivery relation $P$
is a \emph{predicate} over network behaviors in which the eventual delivery
property can hold. In other words, for op-based \CRDTs:
\[
  P \Rightarrow \forall i, j.\, f \in c_i \Rightarrow \lozenge f \in c_j
\]
where it is a standard assumption that $P$ preserves (1) message order up to
concurrent messages and (2) at least once delivery~\citep{shapiro11,
almedia18}.\footnote{In practice, vector timestamps or globally unique
identifiers are associated with each message at the network layer, and messages
are reordered upon delivery to ensure that messages are delivered in the correct
order. Since all messages are eventually delivered under the precondition $P$,
this is a standard assumption.}

However, specifying $\delta$-based \CRDTs as a refinement of state-based \CRDTs
directly would not be sufficient for a constructive proof that $\delta$-state
based \CRDTs achieve \SEC. This is due to the fact that $\delta$-state \CRDTs
send state \emph{fragments}, which makes them the state-based analogue of
op-based \CRDTs. Without an additional delivery relation, $\delta$-state \CRDT
replicas which do not receive some update will never catch up without additional
updates.

Consider the figure below. In this example, we have three $\delta$-state \CRDT
replicas of a $\delta$-based GCounter, and an \textsf{inc} is performed at
$r_1$. Immediately, $r_1$ generates the state fragment $\{ r_1 \mapsto 1 \}$,
and sends it to the other replicas, $r_2$ and $r_3$. For the sake of example,
say that the network drops the update in route to $r_3$ such that it is never
received by $r_3$:

\begin{figure}[H]
  \centering
  \includegraphics[width=.6\textwidth]{figures/sec-delta.pdf}
  \caption{$\delta$-state \CRDTs violating \SEC without the causal merging
    condition.}
  \label{fig:delta-sec-violation}
\end{figure}

Without any future updates, neither of the replicas that \emph{have} received
the update will ever have reason to update $r_3$ again. This is a demonstration
of a \SEC violation, since:
\[
  \{ r_1 \mapsto 1 \} \in r_i \nRightarrow \lozenge \{ r_1 \mapsto 1 \} \in r_3,
  \quad i \in \{ 1, 2 \}
\]
That is, though the update $\{ r_1 \mapsto 1 \}$ is in the node histories of
$r_1$ and $r_2$ (both of which are behaving correctly), that update is
\emph{never} in the history of node $r_3$, which is also behaving correctly.

A critical issue in the above example is that $r_2$ merges the update from $r_1$
immediately--thus placing the update in that node's history--without knowing
whether or not it has been received by $r_3$. An anti-entropy algorithm like
in~\citep{almedia18} addresses these problems. The goal of an anti-entropy
algorithm for $\delta$-state \CRDTs is to do the following:
\begin{enumerate}
  \item On an operation, generate the $\delta$-mutation, and apply it to both
    the local state, and a temporary \emph{$\delta$-group}.
  \item Periodically, randomly choose between the current state and current
    $\delta$-group, and send its entire contents to all other replicas, and
    flush the $\delta$-group.
\end{enumerate}
This ensures that--even without outside interaction--the system as in
Figure~\ref{fig:delta-sec-violation} will eventually recover, since either one
of $r_1$ or $r_2$ will, at some point, send their full state to all other
replicas, including $r_3$, at which point $r_3$ will have caught up.

We believe that it would be a worthwhile research goal to encode this
anti-entropy algorithm into a proof assistant, and specify that $\delta$-state
\CRDTs achieve \SEC \emph{without} a correspondence to traditional op-based
\CRDTs. Similarly to our work, in which we found a correspondence between
$\delta$- and op-based \CRDTs, we believe that specifying $\delta$-state \CRDTs
on their own would highlight the ways in which $\delta$-state \CRDTs are
\emph{different} from op-based \CRDTs.

Likewise, specifying the goal in this fashion would allow the proof to reason
about more network behaviors without a delivery predicate $P$, since the proof
would be aided by the periodic behavior of the anti-entropy algorithm above.

\section{Causally Consistent $\delta$-\CRDTs}

Another difference between op- and state-based \CRDTs is that state-based \CRDTs
require a \emph{causal merging condition} in order to ensure causal consistency
(that is, that updates are applied in a fashion that preserves their causality),
whereas in op-based \CRDTs this is traditionally an assumption placed on
$P$~\citep{shapiro11}.

The authors of~\citep{almedia18} define a \emph{$\delta$-interval}
$\Delta^{a,b}_i$ as:
\[
  \Delta^{a,b}_i = \bigsqcup \left\{ d_i^k : k \in [a, b) \right\}
\]
that is, $\Delta^{a,b}_i$ contains the deltas that occurred at replica $i$
beginning at time $a$ and up until time $b$. They go on to use this
$\delta$-interval to define the \emph{causal merging condition}, which is that
replica $i$ only joins a $\delta$-interval $\Delta^{a,b}_j$ into its own state
$X_i$ if:
\[
  X_i \sqsupseteq X_j^a
\]
That is, updates are only applied locally if they occurred \emph{before} the
latest-known update at replica $i$.

Algorithms which uphold the causal merging condition on $\delta$-intervals have
been proven on paper to satisfy Causal Consistency (\CC) in addition to \SEC. To
our knowledge, this result has not been mechanized, and so we believe it would
be a worthwhile direction of future research to specify the causal merging
condition and associated anti-entropy algorithms which preserve it into an
interactive theorem prover and mechanize the results of~\citep{almedia18}.

If the above is the subject of further exploration, we believe that it would be
additionally possible to prove that $\delta$-state \CRDTs achieve \SEC by a
\emph{simulation proof} that establishes their correspondence with state-based
\CRDTs. This is mentioned as Proposition 3 in~\citep{almedia18}, but we believe
that this is another fruitful area for formal verification.
