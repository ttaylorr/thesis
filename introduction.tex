\chapter{Introduction}

Computational systems today are larger than ever. Whereas previously one would
architect their programs to be run on a single system, it is now commonplace to
design programs that share computation across multiple machines which
communicate with each other in a coordinated fashion. Therefore, it is natural
to ask why one might design from the latter perspective rather than the former.
The answer is threefold:
\begin{enumerate}
  \item \emph{Resiliency}. Designing a computational workload to be distributed
    among participants tolerates the failure of any one (or more) of those
    participants.
  \item \emph{Scalability}. When designed from a distributed standpoint,
    ``scaling'' your workload to meet a higher demand is reduced to adding
    additional hardware, not designing more efficient ways to do the
    computation.
  \item \emph{Locality}. By sharing a computation across many individual pieces
    of hardware, system designers are able to place hardware closer to the site
    at which their requests originate. Suppose that contacting a server capable
    of fulfilling a request bottlenecks the user. In this case, additional
    hardware may be placed physically closer to that user in order to reduce the
    overall latency in responding to a request.
\end{enumerate}

So, it is clear that as our demand on such computations grow, that so too
must our need to design these systems in a way that first considers the concerns
of resiliency, scalability, and locality.

In order to design systems in this way, however, one must consider additionally
the challenges imposed by not having access to shared memory among
participants in the computation. If a program runs in a single-threaded fashion
on a single computer, there is no need to coordinate memory accesses, since only
one part of the program may be reading or writing memory at a given time. If the
program is written to be multithreaded, then the threads must coordinate
among themselves by using mutexes or communication channels to avoid
\textit{race conditions} and other concurrency errors.

The same challenge exists when a system is distributed at the hardware and
machine level, rather than among multiple threads running on a single piece of
hardware. The challenge, however, is made more difficult by the fact that the
communication overhead is far higher between separate pieces of hardware than
between two threads.

This thesis focuses on datatypes by which computation can be coordinated across
multiple machines. In particular, we formalize a set of consistency guarantees
(namely, Strong Eventual Consistency, hereafter \SEC) over a class of replicated
datatypes, $\delta$-Conflict-Free Replicated Datatypes (\CRDTs). We describe the
preliminaries necessary to contextualize the body of this work in the following
section.

\section{Preliminaries}
Our discussion here focuses on a particular class of datatypes that are designed
to be both easily distributed and require relatively low coordination overhead
by allowing individual participants to diverge temporarily from the state of the
overall computation.\footnote{That is, the computation reflects a different
value depending on which participant in the computation responds to the
request.}

These datatypes operate in such a way so as to both avoid conflict between
concurrent updates, and to avoid locking and coordination
overhead~\citep{shapiro11}. \CRDTs are said to achieve \SEC which is to say that
they achieve a stronger form of \textit{eventual consistency} (\EC). We
summarize the definitions of eventual- and strong-eventual consistency
from~\cite{shapiro11}.

\begin{definition}[Eventual Consistency]
  \label{def:eventual-consistency}
  A replicated datatype is \textit{eventually consistent} if:
  \begin{itemize}
    \item Updates delivered to it are eventually delivered to all other replicas
      in the system.
    \item All well-behaved replicas that have received the same set of updates
      eventually reflect the same state.
    \item All executions on this datatype are terminating.
  \end{itemize}
\end{definition}

\begin{definition}[Strong Eventual Consistency]
  A replicated datatype is \textit{strong eventually consistent} if:
  \begin{itemize}
    \item It is eventually consistent, as above.
    \item Convergence occurs immediately, that is, any two replicas that have
      received the same set of updates \textit{always} reflect the same state.
  \end{itemize}
\end{definition}

Broadly speaking, there are two classes of \CRDTs, which we refer to as the op-
and state-based variants. We will provide formal definitions for each of the two
classes in due time (Chapter~\ref{chap:background}), but for now it is
sufficient to distinguish the two as sending different kinds of
messages.\footnote{It is sufficient to consider the op-based \CRDT as sending
representations of actions: increment some value, place an item in a set, and so
on. On the other hand, it suffices to consider state-based \CRDTs as sending the
representation of the effect of that action: ``my counter is now $x$'', ``my set
includes these elements'', and so on. State-based \CRDTs are furthermore
equipped with a $\sqcup$ which is capable of merging two states together, and is
discussed in detail in Section~\ref{sec:state-based-crdts}.}

We present brief definitions of op- and state-based \CRDTs based
on~\citep{baquero14}:

\begin{definition}[Operation-based Conflict-Free Replicated Datatype (op-based
\CRDT)]
  op-based \CRDTs apply updates in two phases:
  \begin{enumerate}
    \item First, an operation is \textit{prepared} locally. At this phase, the
      op-based \CRDT combines the operation with the current state to send a
      representation of the update to other replicas.
    \item Then, the represented operation is applied to other replicas using
      \textit{effect}, where \textit{effect} is commutative for concurrent
      operations.
  \end{enumerate}
\end{definition}

\begin{definition}[State-based Conflict-Free Replicated Datatype (state-based
\CRDT)]
  state-based \CRDTs only apply updates to their local state, and periodically
  send serialized representations of the contents of their state to other
  replicas.

  Crucially, these states form a \textit{monotone join semi-lattice} (i.e,. a
  lattice $\langle S, \sqcup \rangle$ where for any $s_1, s_2 \in S$ at both
  $s_1 \sqsubseteq s_1 \sqcup s_2$ and $s_2 \sqsubseteq s_1 \sqcup s_2$ hold for
  commutative, associative, and idempotent $\sqcup$).

  To achieve convergence, state-based \CRDTs periodically send their state to
  other replicas, which then replace their own state by joining the received
  state into their own.
\end{definition}

\section{op- and state-based trade-offs}

These two classes are distinguished from one another based on their strengths
and weaknesses. In one sense, op- and state-based \CRDTs form a kind of a dual,
where they trade off strong network guarantees for message payload
size~\citep{baquero14}.

Because the state-based \CRDT needs to send a representation of its entire state,
it often requires a significant amount of network bandwidth to propagate large
messages~\citep{almedia18}. For example, in the state-based vector counter
(which we will describe in detail in Section~\ref{sec:example-gcounter}), the
payload size grows as a linear function of the number of replicas. In return for
this large payload size, state-based \CRDTs are able to achieve strong eventual
consistency even in network that are allowed to drop, reorder, and duplicate
messages.

On the other hand, op-based \CRDTs require relatively little network bandwidth
to send a notification of a single update (typically the representation
generated in the \textit{prepare} stage is dwarfed by the typical payload size
of a state-based \CRDT), but in exchange demand that the network deliver
messages in-order for sequential (comparable) updates and
at-most-once~\citep{shapiro11}.

Significant work in this area~\cite{almedia18, enes18, cabrita17, vanDerLinde16}
has focused on mediating these two extremes. This line of research (particularly
in~\citep{almedia18}) has identified $\delta$-state \CRDTs---a variant of the
state-based \CRDT which we discuss in Section~\ref{sec:state-based-crdts}---as
an alternative which occupies a satisfying position between the two extremes.
$\delta$-state \CRDTs behave as traditional state-based \CRDTs, with the
exception that their updates consist of state \emph{fragments} instead of their
entire state. These fragments (generated by $\delta$-mutators and called
$\delta$-updates) are then applied locally at all other replicas to reassemble
the full state. Because these fragments often do not need to comprise the full
state, $\delta$-state \CRDTs have in general, small payload size (thus requiring
a similar amount of bandwidth as messages sent and received from op-based
\CRDTs), while still tolerating the same set of network deficiencies as
state-based \CRDTs. This combination of properties makes them an appealing
alternative to traditional state- and op-based \CRDTs, and places interest in
studying their convergence properties.

\section{Contributions}

Our main contribution builds on the work in~\citep{gomes17} and introduces a set
of formally verified proofs in Isabelle~\citep{wenzel02} that re-establish the
main result of~\citep{almedia18}, which we re-state below:

\begin{theorem}[Almedia, Shoker, Baquero, '18]
  Consider a set of replicas of a $\delta$-\CRDT object, replica $i$ evolving
  along a sequence of states $X_i^0 = \bot$, $X_i^1=\ldots$, , each replica
  performing delta-mutations of the form $m^\delta_{i,k}(X^k_i)$ at some subset
  of its sequence of states, and evolving by joining the current state either
  with self-generated deltas or with delta-groups received from others. If each
  delta-mutation $m^\delta _{i,k}(X^k_i)$ produced at each replica is joined
  (directly or as part of a delta-group) at least once with every other replica,
  all replica states become equal.
\end{theorem}

Here, $X_i^t$ refers to the state of replica $i$ at time $t$, and
$m^\delta_{i,k}$ refers to the $\delta$-mutation applied at the $i$th replica
$i$ at time $k$.

In other words, we rely on the work of~\citep{gomes17} in order to build a
portfolio of $\delta$-state \CRDTs as in~\citep{almedia18} and show that even
under weak network guarantees\footnote{We inherit dropping and reordering of
messages from the original work of~\citep{gomes17}, but further weaken the
network model by also allowing messages to be duplicated.} these $\delta$-state
\CRDTs still achieve strong eventual consistency.

Our verification efforts yielded a portfolio of different $\delta$-state \CRDTs
(the grow-only counter (GCounter) and set (GSet), as well as composite variants
such as the positive-negative counter (PN-Counter), and the observed-remove set
(OR-set)). Our key idea guiding these verification efforts is to treat op- and
state-based \CRDTs similarly by modeling state-based \CRDTs as op-based where the
operation is the join provided by the semi-lattice.\footnote{This approach is
described in detail in Section~\ref{sec:state-as-op}.} We use this idea to
build state- and $\delta$-state \CRDTs on top of the library provided
in~\citep{gomes17}, and show that the
$strong{\isacharunderscore}eventual{\isacharunderscore}consistency$ locale can
still be instantiated over this portfolio, even when the underlying $network$
locale has been weakened substantially from when it was introduced in the
aforementioned work.

The reminder of this thesis is ordered as follows:
\begin{itemize}
  \item In Chapter~\ref{chap:background}, we summarize existing research in the
    broader realm of conflict-free replicated datatypes. We present formal
    definitions of op- and state-based \CRDTs, and conduct a thorough discussion
    of their relative strengths and weaknesses. Likewise, we present a summary
    of some work in the area of $\delta$-state \CRDTs, and present its strengths.
  \item In Chapter~\ref{chap:overview}, we outline our specific research goals,
    as well as the work in~\citep{almedia18} upon which much of our work is
    built. We lay out our specific approach, as well as some of the key insights
    that guided our proof strategy.
  \item In Chapter~\ref{chap:results}, we discuss the outcome of our approach by
    presenting a portfolio of successfully-verified $\delta$-state \CRDTs, as
    well as describe our efforts in weakening the network model in order to
    verify these objects over a non-trivial set of network behaviors.
  \item We conclude in Chapter~\ref{chap:future-work} by suggesting future
    research directions. We suggest some possible areas where formalizing some
    of our proofs may be fruitful, as well as future approaches that would be of
    interest to the $\delta$-state \CRDT research community.
\end{itemize}
