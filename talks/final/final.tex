\ifdefined\handout
  \documentclass[aspectratio=169,compress,handout]{beamer}
\else
  \documentclass[aspectratio=169,compress]{beamer}
\fi

\usepackage{amsfonts}
\usepackage{amsmath}
\usepackage{amssymb}
\usepackage{amsthm}
\usepackage[square,sort,comma,numbers]{natbib}
\usepackage{xcolor}
\usepackage{xspace}

\newcommand*{\TODO}[1][]{\text{\color{red} TODO {#1}}}

\newcommand*{\CRDT}{\textsf{CRDT}\xspace}
\newcommand*{\CRDTs}{\textsf{CRDTs}\xspace}
\newcommand*{\CC}{\textsf{CC}\xspace}
\newcommand*{\EC}{\textsf{EC}\xspace}
\newcommand*{\SEC}{\textsf{SEC}\xspace}

\setbeamercovered{transparent}


\bibliography{bibliography.bib}

\title{Verifying Strong Eventual Consistency in $\delta$-\CRDTs}
\author{Taylor Blau}
\institute{University of Washington}
\date{June 2020}

\begin{document}
  \frame{\titlepage}

  \begin{frame}
    \frametitle{Contributions}

    This thesis:
    \begin{itemize}[<+->]
      \item Mechanized proofs in Isabelle that two $\delta$-state \CRDTs inhabit
        \SEC.
        \begin{itemize}
          \item Reuse a library for verifying operation-based \CRDTs of Gomes to
            reason about $\delta$-state \CRDTs.
          \item Weaken the network model of Gomes' to support duplicated
            messages.
        \end{itemize}
      \item Two reductions that allow us to reason about $\delta$-state \CRDTs
        in terms of operation-based \CRDTs.
      \item Two encodings of the latter reduction.
    \end{itemize}
  \end{frame}

  \begin{frame}
    \frametitle{This talk}

    \begin{itemize}[<+->]
      \item Why distributed systems?
      \item Consistency models: classic approaches and relaxed approximations.
      \item \CRDTs: operation-, state- and $\delta$-state based, and the
        trade-offs each makes.
      \item Reductions between \CRDT variants.
      \item Mechanized proofs in two encodings.
      \item Conclusion.
    \end{itemize}
  \end{frame}

  \begin{frame}
    Why distributed systems?
    \begin{enumerate}
      \item \emph{Resiliency}. Tolerates failure of any one (or more)
        participants.
      \item \emph{Scalability}. Meeting the demands of an increased workload as
        simple as adding more hardware.
      \item \emph{Locality}. Service requests to varied locations by placing
        hardware close to where requests originate.
    \end{enumerate}
  \end{frame}
\end{document}
